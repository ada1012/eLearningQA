\documentclass[a4paper,12pt,twoside]{memoir}

% Castellano
\usepackage[spanish,es-tabla]{babel}
\usepackage{graphicx}
\usepackage{float}
\selectlanguage{spanish}
\usepackage[utf8]{inputenc}
\usepackage[T1]{fontenc}
\usepackage{lmodern} % Scalable font
\usepackage{microtype}
\usepackage{placeins}

\usepackage{url}

\RequirePackage{booktabs}
\RequirePackage[table]{xcolor}
\RequirePackage{xtab}
\RequirePackage{multirow}

% Links
\PassOptionsToPackage{hyphens}{url}\usepackage[colorlinks]{hyperref}
\hypersetup{
	allcolors = {red}
}

% Bibliography management
\usepackage[numbers,sort]{natbib}

% Ecuaciones
\usepackage{amsmath}

% Rutas de fichero / paquete
\newcommand{\ruta}[1]{{\sffamily #1}}

% Párrafos
\nonzeroparskip

% Huérfanas y viudas
\widowpenalty100000
\clubpenalty100000

% Imagenes
\usepackage{graphicx}
\newcommand{\imagen}[2]{
	\begin{figure}[!h]
		\centering
		\includegraphics[width=0.9\textwidth]{#1}
		\caption{#2}\label{fig:#1}
	\end{figure}
	\FloatBarrier
}

\newcommand{\imagenflotante}[2]{
	\begin{figure}%[!h]
		\centering
		\includegraphics[width=0.9\textwidth]{#1}
		\caption{#2}\label{fig:#1}
	\end{figure}
}



% El comando \figura nos permite insertar figuras comodamente, y utilizando
% siempre el mismo formato. Los parametros son:
% 1 -> Porcentaje del ancho de página que ocupará la figura (de 0 a 1)
% 2 --> Fichero de la imagen
% 3 --> Texto a pie de imagen
% 4 --> Etiqueta (label) para referencias
% 5 --> Opciones que queramos pasarle al \includegraphics
% 6 --> Opciones de posicionamiento a pasarle a \begin{figure}
\newcommand{\figuraConPosicion}[6]{%
  \setlength{\anchoFloat}{#1\textwidth}%
  \addtolength{\anchoFloat}{-4\fboxsep}%
  \setlength{\anchoFigura}{\anchoFloat}%
  \begin{figure}[#6]
    \begin{center}%
      \Ovalbox{%
        \begin{minipage}{\anchoFloat}%
          \begin{center}%
            \includegraphics[width=\anchoFigura,#5]{#2}%
            \caption{#3}%
            \label{#4}%
          \end{center}%
        \end{minipage}
      }%
    \end{center}%
  \end{figure}%
}

%
% Comando para incluir imágenes en formato apaisado (sin marco).
\newcommand{\figuraApaisadaSinMarco}[5]{%
  \begin{figure}%
    \begin{center}%
    \includegraphics[angle=90,height=#1\textheight,#5]{#2}%
    \caption{#3}%
    \label{#4}%
    \end{center}%
  \end{figure}%
}
% Para las tablas
\newcommand{\otoprule}{\midrule [\heavyrulewidth]}
%
% Nuevo comando para tablas pequeñas (menos de una página).
\newcommand{\tablaSmall}[5]{%
 \begin{table}
  \begin{center}
   \rowcolors {2}{gray!35}{}
   \begin{tabular}{#2}
    \toprule
    #4
    \otoprule
    #5
    \bottomrule
   \end{tabular}
   \caption{#1}
   \label{tabla:#3}
  \end{center}
 \end{table}
}

%
% Nuevo comando para tablas pequeñas (menos de una página).
\newcommand{\tablaSmallSinColores}[5]{%
 \begin{table}[H]
  \begin{center}
   \begin{tabular}{#2}
    \toprule
    #4
    \otoprule
    #5
    \bottomrule
   \end{tabular}
   \caption{#1}
   \label{tabla:#3}
  \end{center}
 \end{table}
}

\newcommand{\tablaApaisadaSmall}[5]{%
\begin{landscape}
  \begin{table}
   \begin{center}
    \rowcolors {2}{gray!35}{}
    \begin{tabular}{#2}
     \toprule
     #4
     \otoprule
     #5
     \bottomrule
    \end{tabular}
    \caption{#1}
    \label{tabla:#3}
   \end{center}
  \end{table}
\end{landscape}
}

%
% Nuevo comando para tablas grandes con cabecera y filas alternas coloreadas en gris.
\newcommand{\tabla}[6]{%
  \begin{center}
    \tablefirsthead{
      \toprule
      #5
      \otoprule
    }
    \tablehead{
      \multicolumn{#3}{l}{\small\sl continúa desde la página anterior}\\
      \toprule
      #5
      \otoprule
    }
    \tabletail{
      \hline
      \multicolumn{#3}{r}{\small\sl continúa en la página siguiente}\\
    }
    \tablelasttail{
      \hline
    }
    \bottomcaption{#1}
    \rowcolors {2}{gray!35}{}
    \begin{xtabular}{#2}
      #6
      \bottomrule
    \end{xtabular}
    \label{tabla:#4}
  \end{center}
}

%
% Nuevo comando para tablas grandes con cabecera.
\newcommand{\tablaSinColores}[6]{%
  \begin{center}
    \tablefirsthead{
      \toprule
      #5
      \otoprule
    }
    \tablehead{
      \multicolumn{#3}{l}{\small\sl continúa desde la página anterior}\\
      \toprule
      #5
      \otoprule
    }
    \tabletail{
      \hline
      \multicolumn{#3}{r}{\small\sl continúa en la página siguiente}\\
    }
    \tablelasttail{
      \hline
    }
    \bottomcaption{#1}
    \begin{xtabular}{#2}
      #6
      \bottomrule
    \end{xtabular}
    \label{tabla:#4}
  \end{center}
}

%
% Nuevo comando para tablas grandes sin cabecera.
\newcommand{\tablaSinCabecera}[5]{%
  \begin{center}
    \tablefirsthead{
      \toprule
    }
    \tablehead{
      \multicolumn{#3}{l}{\small\sl continúa desde la página anterior}\\
      \hline
    }
    \tabletail{
      \hline
      \multicolumn{#3}{r}{\small\sl continúa en la página siguiente}\\
    }
    \tablelasttail{
      \hline
    }
    \bottomcaption{#1}
  \begin{xtabular}{#2}
    #5
   \bottomrule
  \end{xtabular}
  \label{tabla:#4}
  \end{center}
}



\definecolor{cgoLight}{HTML}{EEEEEE}
\definecolor{cgoExtralight}{HTML}{FFFFFF}

%
% Nuevo comando para tablas grandes sin cabecera.
\newcommand{\tablaSinCabeceraConBandas}[5]{%
  \begin{center}
    \tablefirsthead{
      \toprule
    }
    \tablehead{
      \multicolumn{#3}{l}{\small\sl continúa desde la página anterior}\\
      \hline
    }
    \tabletail{
      \hline
      \multicolumn{#3}{r}{\small\sl continúa en la página siguiente}\\
    }
    \tablelasttail{
      \hline
    }
    \bottomcaption{#1}
    \rowcolors[]{1}{cgoExtralight}{cgoLight}

  \begin{xtabular}{#2}
    #5
   \bottomrule
  \end{xtabular}
  \label{tabla:#4}
  \end{center}
}



\graphicspath{ {./img/} }

% Capítulos
\chapterstyle{bianchi}
\newcommand{\capitulo}[2]{
	\setcounter{chapter}{#1}
	\setcounter{section}{0}
	\setcounter{figure}{0}
	\setcounter{table}{0}
	\chapter*{#2}
	\addcontentsline{toc}{chapter}{#2}
	\markboth{#2}{#2}
}

% Apéndices
\renewcommand{\appendixname}{Apéndice}
\renewcommand*\cftappendixname{\appendixname}

\newcommand{\apendice}[1]{
	%\renewcommand{\thechapter}{A}
	\chapter{#1}
}

\renewcommand*\cftappendixname{\appendixname\ }

% Formato de portada
\makeatletter
\usepackage{xcolor}
\newcommand{\tutor}[1]{\def\@tutor{#1}}
\newcommand{\course}[1]{\def\@course{#1}}
\definecolor{cpardoBox}{HTML}{E6E6FF}
\def\maketitle{
  \null
  \thispagestyle{empty}
  % Cabecera ----------------
\noindent\includegraphics[width=\textwidth]{cabecera}\vspace{1cm}%
  \vfill
  % Título proyecto y escudo informática ----------------
  \colorbox{cpardoBox}{%
    \begin{minipage}{.8\textwidth}
      \vspace{.5cm}\Large
      \begin{center}
      \textbf{TFG del Grado en Ingeniería Informática}\vspace{.6cm}\\
      \textbf{\LARGE\@title{}}
      \end{center}
      \vspace{.2cm}
    \end{minipage}

  }%
  \hfill\begin{minipage}{.20\textwidth}
    \includegraphics[width=\textwidth]{escudoInfor}
  \end{minipage}
  \vfill
  % Datos de alumno, curso y tutores ------------------
  \begin{center}%
  {%
    \noindent\LARGE
    Presentado por \@author{}\\ 
    en Universidad de Burgos --- \@date{}\\
    Tutores: \@tutor{}\\
  }%
  \end{center}%
  \null
  \cleardoublepage
  }
\makeatother

\newcommand{\nombre}{Alberto Díaz Álvarez} %%% cambio de comando

% Datos de portada
\title{eLearningQA Versión 2\\- Cuestionarios y Foros -}
\author{\nombre}
\tutor{Carlos López Nozal y Raúl Marticorena Sánchez}
\date{\today}

\begin{document}

\maketitle


\newpage\null\thispagestyle{empty}\newpage


%%%%%%%%%%%%%%%%%%%%%%%%%%%%%%%%%%%%%%%%%%%%%%%%%%%%%%%%%%%%%%%%%%%%%%%%%%%%%%%%%%%%%%%%
\thispagestyle{empty}


\noindent\includegraphics[width=\textwidth]{cabecera}\vspace{1cm}

\noindent D. Carlos López Nozal, profesor del departamento de Ingeniería Informática,
área de Lenguajes y Sistemas Informáticos.

\noindent D. Raúl Marticorena Sánchez, profesor del departamento de Ingeniería Informática,
área de Lenguajes y Sistemas Informáticos.

\noindent Exponen:

\noindent Que el alumno D. \nombre, con DNI 03929601M, ha realizado el Trabajo final de Grado en Ingeniería Informática titulado \textit{eLearningQA Versión 2 - Cuestionarios y Foros}. 

\noindent Y que dicho trabajo ha sido realizado por el alumno bajo la dirección, en virtud de lo cual se autoriza su presentación y defensa.

\begin{center} %\large
En Burgos, {\large \today}
\end{center}

\vfill\vfill\vfill

% Author and supervisor
\begin{minipage}{0.45\textwidth}
\begin{flushleft} %\large
Vº. Bº. del Tutor:\\[2cm]
D. Raúl Marticorena Sánchez
\end{flushleft}
\end{minipage}
\hfill
\begin{minipage}{0.45\textwidth}
\begin{flushleft} %\large
Vº. Bº. del tutor:\\[2cm]
D. Carlos López Nozal
\end{flushleft}
\end{minipage}
\hfill

\vfill

% para casos con solo un tutor comentar lo anterior
% y descomentar lo siguiente
%Vº. Bº. del Tutor:\\[2cm]
%D. nombre tutor


\newpage\null\thispagestyle{empty}\newpage




\frontmatter

% Abstract en castellano
\renewcommand*\abstractname{Resumen}
\begin{abstract}
Durante los últimos años la enseñanza ha sentido la obligación de un traslado paulatino hacia la docencia online permitiendo la realización de grados a distancia. Esto ha implicado adaptar los métodos de enseñanza y evaluación en los entornos virtuales de aprendizaje como Moodle. La calidad, en el contexto de enseñanza online, se refiere a la capacidad de cumplir con las necesidades educativas del alumno.  Este trabajo tiene como objetivo evaluar la calidad de los recursos y actividades de los cursos implementados en Moodle y proporcionar posibles acciones de mejora. En este TFG se utiliza MOOQ (Massive Open Online Quality) como el conjunto de estándares y directrices para evaluar y promover la calidad de los cursos en línea.  Este marco de trabajo proporciona distintas perspectivas (pedagógica, tecnológica y estratégica), identificando roles en el curso (diseñador, facilitador y proveedor) y las diferentes fases (análisis, diseño, implementación y evaluación). Los resultados de este TFG se basan en la incorporación de nuevas reglas de calidad, siguiendo los estándares y las directrices del marco de trabajo, a un TFG previo. Las nuevas reglas de calidad consultan el uso de las actividades de foros y cuestionarios de un curso Moodle. Se ha conseguido complementar el trabajo previo, de generación de un informe de calidad de un curso Moodle, incorporando reglas de participación en cuestionarios y foros, reglas de validación de resultados de cuestionarios y reglas de análisis de sentimientos en las intervenciones en los foros.


\end{abstract}

\renewcommand*\abstractname{Descriptores}
\begin{abstract}
Aplicación web, Cuestionarios, Foros, Estadísticas, Participación.
\end{abstract}

% Abstract en inglés
\renewcommand*\abstractname{Abstract}
\begin{abstract}
In recent years, education has felt the obligation to gradually transition towards online teaching, allowing for the completion of degrees remotely. This has involved adapting teaching and evaluation methods in virtual learning environments such as Moodle. Quality, in the context of online education, refers to the ability to meet the educational needs of the student. The objective of this work is to evaluate the quality of resources and activities in courses implemented on Moodle and provide potential areas for improvement. In this Bachelor's thesis, MOOQ (Massive Open Online Quality) is used as the set of standards and guidelines to assess and promote the quality of online courses. This framework provides different perspectives (pedagogical, technological, and strategic), identifying roles in the course (designer, facilitator, and provider) and different phases (analysis, design, implementation, and evaluation). The results of this Bachelor's thesis are based on the incorporation of new quality rules, following the standards and guidelines of the framework, into a previous thesis. The new quality rules address the use of forum activities and quizzes in a Moodle course. The previous work, which generated a quality report for a Moodle course, has been complemented by incorporating rules for participation in quizzes and forums, rules for validating quiz results, and rules for sentiment analysis in forum interactions.
\end{abstract}

\renewcommand*\abstractname{Descriptores}
\begin{abstract}
Web application, Quizzes, Forums, Statistics, Participation.
\end{abstract}

\clearpage

% Indices
\tableofcontents

\clearpage

\listoffigures

\clearpage

\listoftables
\clearpage

\mainmatter
\capitulo{1}{Introducción}

En el presente Trabajo de Fin de Grado (TFG), se aborda el desarrollo de una aplicación innovadora que se centra en el aseguramiento de la calidad en el ámbito del e-learning. El e-learning, o aprendizaje electrónico, ha experimentado un crecimiento significativo en los últimos años, convirtiéndose en una alternativa educativa cada vez más popular y relevante en diversos entornos académicos y corporativos.

Sin embargo, a medida que el e-learning se ha expandido, también han surgido desafíos en términos de asegurar la calidad de los contenidos y las experiencias de aprendizaje ofrecidas. Es fundamental garantizar que los recursos educativos en línea sean efectivos, accesibles, relevantes y estén diseñados de acuerdo con los estándares y las mejores prácticas establecidas.

El objetivo principal de esta aplicación es proporcionar a los diseñadores, desarrolladores y responsables de la calidad en el e-learning una herramienta integral para evaluar y mejorar la calidad de los cursos en línea. La aplicación se enfoca en varios aspectos clave del aseguramiento de la calidad, incluyendo el diseño instruccional, la usabilidad, la accesibilidad, la interactividad y la evaluación del aprendizaje. 

La aplicación permitirá a los usuarios realizar evaluaciones sistemáticas de los cursos en línea, identificando fortalezas y áreas de mejora en cada aspecto relevante. Además, ofrecerá pautas y recomendaciones prácticas para mejorar la calidad de los cursos, ayudando a los profesionales del e-learning a desarrollar experiencias de aprendizaje más efectivas y satisfactorias.

A lo largo de este trabajo, se describirá en detalle el proceso de desarrollo de la aplicación, desde el diseño de la arquitectura y la interfaz de usuario, hasta la implementación de las funcionalidades clave. Se abordarán los desafíos técnicos y las decisiones de diseño tomadas, así como las pruebas y validaciones realizadas para garantizar el correcto funcionamiento de la aplicación.

Además, se explorarán y analizarán las normativas, estándares y mejores prácticas relevantes en el ámbito del aseguramiento de la calidad en el e-learning, con el fin de fundamentar y respaldar las funcionalidades y recomendaciones proporcionadas por la aplicación.
\capitulo{2}{Objetivos del proyecto}

El objetivo principal de este trabajo es continuar con el desarrollo de una aplicación web que permita al profesorado evaluar las distintas fases de diseño instruccional de un curso de Moodle (diseño, implementación, realización, evaluación), tal como recomiendan algunos frameworks internacionales de calidad en e-learning \cite{previotfg}. Este trabajo se centrará principalente en las fases de diseño y realización.

Para cumplir dicho objetivo se ha decidido profundizar en el apartado de los cuestionarios y los foros intentando conseguir un informe detallado en dichas fases. Consiguiendo así una rápida lectura de la viabilidad del curso con la posibilidad de obtener detalles en las posibles zonas de mejora.

A continuación se detallarán los subobjetivos que darán pie al cumplimiento del objetivo principal:
\begin{enumerate}
    \item Definir los modelos y sus respectivos atributos de los cuestionarios y foros.
    \item Combinar los servicios Web de Moodle para adaptar los datos obtenidos a las nuevas funcionalidades.
    \item Adaptar la información de los cuestionarios y foros a la interfaz ya creada para mantener un aspecto amigable y funcional.
    \item Aplicar los frameworks internacionales de calidad en e-learning a los datos que reproducirá el proyecto.
    \item Diseñar indicadores cualitativos y cuantitativos de calidad de cada fase de diseño instruccional del curso en línea (diseño, implementación, realización y evaluación) \cite{previotfg}.
\end{enumerate}
\capitulo{3}{Conceptos teóricos}


\section{Definiciones básicas}


\capitulo{4}{Técnicas y herramientas}
\section{Desarrollo ágil}
Durante la realización de esta segunda versión se ha mantenido la metodogía de desarrollo ágil siguiendo la línea de una evolución constante, permitiendo la obtención del feedback entre alumno y profesor de forma continua.
\subsection{Desarrollo iterativo}
El desarrollo consiste en la revisión cíclica sobre un mismo trabajo. En este caso se han realizado sprints de entre 7 y 14 días en los que se establecía una reunión en la plataforma Microsoft Teams al final del sprint mostrando los resultados y recibiendo una retroalimentación de los errores, posibles mejoras y características interesantes de implementar de cara al siguiente sprint.
\subsection{Desarrollo incremental}
El desarrollo incremental sigue la dinámica del desarrollo iterativo buscando la continua mejora gracias al feedback obtenido. Con este procedimiento se han ido realizando varias release a lo largo del trabajo. Una release es una nueva versión del sistema que se está desarrollando\cite{releasedefinition}.
\subsection{Control de versiones}
El control de versiones es un sistema que se utiliza para gestionar y controlar los cambios realizados en los archivos y documentos de un proyecto o sistema de software a lo largo del tiempo. Permite realizar un seguimiento de las modificaciones, controlar quién ha realizado cada cambio, revertir a versiones anteriores y colaborar de manera efectiva en el desarrollo de software.

El control de versiones es especialmente importante en el desarrollo de software, donde múltiples personas trabajan en el mismo proyecto y realizan cambios en los archivos de código fuente. Con un sistema de control de versiones, los desarrolladores pueden guardar y compartir las versiones de su código, fusionar los cambios realizados por diferentes personas y resolver conflictos que puedan surgir.
\subsection{GitHub}
GitHub es la plataforma web de alojamiento y colaboración para proyectos de desarrollo de software que se ha utilizado para este proyecto. Proporciona un sistema de control de versiones distribuido utilizando Git y ofrece herramientas y funcionalidades adicionales para la gestión de proyectos y el trabajo en equipo.
\subsection{Extensión de Visual Studio Code - Git Graph}
Git Graph es una extensión del editor de texto Visual Studio Code que proporciona una interfaz gráfica interactiva para visualizar y navegar por la historia de un repositorio de Git. Permite a los desarrolladores ver de manera intuitiva el historial de cambios, las ramas, las fusiones y las etiquetas de un proyecto Git.

Esta extensión muestra un gráfico visual en forma de árbol que representa la estructura del historial de commits del repositorio. Cada commit se muestra como un nodo en el gráfico y las líneas conectan los nodos para mostrar la relación entre ellos. Además, Git Graph proporciona información adicional sobre cada commit, como el autor, el mensaje de commit y la fecha.\imagen{GitGraph.png}{Extensión de Visual Studio Code - Git Graph}
\subsection{Publicaciones frecuentes}
Una publiación o release consiste en desplegar una aplicación funcional que permite la interacción de los usuarios con dicho producto. Durante el desarrollo del proyecto se han publicado 3 versiones.
\subsection{Refactorización}
La refactorización es el proceso de modificar el diseño interno de un código fuente sin cambiar su comportamiento externo. Se trata de mejorar la estructura y la calidad del código sin añadir nuevas funcionalidades o alterar su funcionalidad existente. El objetivo principal de la refactorización es hacer que el código sea más legible, mantenible y eficiente. En este proyecto se ha implementado SonarCloud para realizar dichas refactorizaciones.
\subsection{Uso de test unitarios}
Un test unitario es una técnica de pruebas en el desarrollo de software que tiene como objetivo verificar el correcto funcionamiento de una unidad de código, por lo general, una función, método o clase, de forma aislada e independiente del resto del sistema. En esta aplicación se ha utilizado jUnit para ejecutar los test.
\subsubsection{JUnit}
JUnit es un framework para la realización de tests en Java. Es compatible con varios entornos de desarrollo integrados e incluso se puede usar mediante linea de comandos.
\subsection{Construcción automática}
La construcción automática es el uso de herramientas como Maven o Gradle para automatizar procesos como la compilación, la ejecución de tests y el empaquetado del software.
\subsubsection{Maven}
Maven es una herramienta para la construcción de proyectos software. Utiliza un archivo definido dentro del proyecto llamado ``pom.xml'' para definir la configuración necesaria para construir el proyecto como las dependencias o el formato al que compilar.
\subsection{Integración continua}
La integración continua  es una práctica de desarrollo de software que consiste en fusionar y probar los cambios realizados en el código fuente de forma frecuente y automatizada. 
El proyecto contiene un archivo yml que especifica las acciones a seguir para la construcción automática del proyecto y el análisis de calidad de SonarCloud tras cada \textit{push} a la rama \textit{develop} del repositorio.
\subsection{Herramienta de calidad de código: SonarCloud}
SonarCloud es un servicio en la nube de análisis de código que detecta code smells, bugs, y vulnerabilidades de seguridad. Se encuentra integrado en el ciclo de integración continua. Permite definir un ``Quality gate'' para que la integración continua falle en caso de no cumplirse alguna de las condiciones definidas sobre la calidad del código.
También, al ser SonarCloud una herramienta de control de calidad, ha servido de inspiración al implementar la lista de aspectos a mejorar incluida en los informes que genera la aplicación.
Se puede ver la evolución de la calidad del código en la figura \ref{fig:sonarcloud} y en \url{https://sonarcloud.io/project/activity?category=QUALITY_GATE&id=ada1012_eLearningQA}.

\section{Herramientas de desarrollo}
\subsection{Entorno de desarrollo integrado: Eclipse y Visual Studio Code}
Por motivos personales en este trabajo se ha trabajado con 2 entornos de desarrollo, Eclipse y Visual Studio Code. Esta decisión se ha tomado por la comodidad de configuración de archivos de Eclipse y la interfaz amigable de Visual Studio Code. Este último permite instalar varias extensiones como Git Graph o GitHub Copilot que hacen la programación mucho más amena.
\subsection{Extensión de Visual Studio Code - GitHub Copilot}
GitHub Copilot es una herramienta desarrollada por GitHub y OpenAI que utiliza inteligencia artificial (IA) para proporcionar sugerencias y autocompletar código mientras se escribe en diferentes lenguajes de programación. Funciona como una extensión para el editor de código y aprovecha los modelos de lenguaje generativos de OpenAI, como GPT-3, para ofrecer recomendaciones de código en tiempo real.
Esto permite obtener recomendaciones de código sin necesidad de acudir al navegador.
\subsection{Framework CSS: Bootstrap}
Bootstrap es un conjunto de librerías de CSS y javascript de código abierto creado por empleados de Twitter. Ofrece cantidad de recursos que facilitan la disposición de elementos en pantalla y contiene elementos como acordeones y carruseles ya implementados.
\subsection{Librería de generación de gráficos: Plotly}
Plotly es una librería de generación de gráficos para el lenguaje de programación Python que también tiene una versión para Javascript. Se ha elegido por estar recomendada por W3Schools, una página web de tutoriales de desarrollo web creada en 1998, y por ser capaz de dibujar líneas independientes entre si al usar pares de coordenadas \textit{x} e \textit{y} y no tablas de valores.

\section{Herramientas de documentación}
\subsection{Redacción de memoria y anexos: Overleaf}
Overleaf es una herramienta en línea de colaboración y edición de documentos LaTeX. Permite a los usuarios crear, editar y compilar documentos LaTeX en un entorno en la nube, sin necesidad de instalar software adicional. Overleaf es especialmente popular entre la comunidad académica y científica, ya que facilita la colaboración en tiempo real y simplifica el proceso de escritura de documentos técnicos y científicos.
\subsection{Generación de tablas: TablesGenerator.com}
TablesGenerator.com es una página que permite crear una tabla con facilidad y puede convertirla a distintos formatos (LaTeX, HTML, texto simple...). Se ha utilizado para diseñar las tablas utilizadas en la aplicación y las tablas en memoria y anexos.
\subsection{Generación de diagramas UML: ArgoUML}
ArgoUML es una herramienta de dibujo de diagramas UML. Se ha utilizado para todos los diagramas de los anexos. Se ha utilizado para realizar diagramas de clases, de secuencia, y de casos de uso.

\section{Patrón de diseño: Fachada}
El patrón de diseño fachada consiste en crear una clase que haga de intermediario entre el cliente y uno o varios subsistemas de la aplicación con varios propósitos: Simplificar y centralizar el control, actuar como elemento de seguridad restringiendo el acceso a ciertas partes, y separar responsabilidades de los subsistemas. Un mismo sistema podría tener varias fachadas distintas que den un mismo servicio de distintas formas, por ejemplo, en este caso, la fachada se utiliza para generar parte del contenido de la aplicación web, pero si se quisiera trasladar la aplicación a una aplicación de escritorio solo se tendría que crear una nueva fachada dejando los sistemas subyacentes intactos \cite{gamma1995design}. Para más información sobre el uso concreto que se le da a este patrón de diseño, consultar el apartado C.4 de los anexos.

 \section{Herramientas para acceder a la información}
En este apartado se va a explicar cómo se ha accedido a la información de los cursos de Moodle y las opciones que se han barajado durante las reuniones con los tutores.
\subsection{Opción elegida - Web services}
Web services son conjuntos de protocolos y estándares que permiten la comunicación y el intercambio de datos entre diferentes aplicaciones o sistemas a través de la web. Utilizando HTTP como protocolo de transporte, los web services permiten que las aplicaciones se comuniquen y compartan información de manera interoperable e independiente de la plataforma.

Los web services de Moodle, un sistema de gestión de aprendizaje en línea, proporcionan interfaces de programación (API) que permiten la integración de Moodle con otras aplicaciones. Estos web services permiten realizar operaciones como la autenticación de usuarios, el acceso a recursos y actividades del curso y la obtención de datos relacionados con los usuarios, cursos y calificaciones.
\subsection{Opción planteada - Web scraping}
Web scraping es una técnica automatizada que consiste en extraer y recopilar datos de manera estructurada de sitios web. Esta técnica implica el uso de software o scripts para acceder a las páginas web, analizar su contenido y extraer la información deseada, como texto, imágenes, enlaces u otros datos relevantes.
Al final no se eligió esta opción ya que la mayoría de información que requería la aplicación ya la ofrecía Moodle a través de sus web services.

\section{Framework de desarrollo web}
En este apartado se ha heredado el framework ya implementado Spring.
\subsection{Spring}
Spring es un framework de desarrollo de aplicaciones empresariales para la plataforma Java. Proporciona una infraestructura completa y coherente que facilita la creación de aplicaciones escalables y de alta calidad. Spring se basa en los principios de inversión de control (IoC) y la inyección de dependencias (DI), lo que permite una mayor modularidad, flexibilidad y facilita las pruebas unitarias. Además, Spring ofrece una amplia gama de módulos y funcionalidades que abarcan desde la creación de servicios web hasta la integración con bases de datos y la seguridad.
\capitulo{5}{Aspectos relevantes del desarrollo del proyecto}
\section{Ciclo de vida}
Este trabajo se ha realizado como se ha comentado previamente con la implementación de sprints de 14 días al principio del trabajo y luego se redujeron las reuniones a un período de 7 días ya que el tiempo jugaba un papel importante y faltaban muchas características por implementar.

Esta segunda versión comenzó con la comprensión del proyecto heredado y el planteamiento de si se iba a mantener la aplicación web programada con Java o si se iba a trasladar a un lenguaje de programación diferente. Gracias a los conocimientos adquiridos previamente en un grado superior de desarrollo de aplicaciones web ya se partía con una experiencia previa en JSP por lo que se mantuvo el proyecto intacto.
Después se decidió trabajar en la profundización de conocimientos en el área de cuestionarios y foros como indica el título del TFG. En esta especialización, primero se trabajó con el tema de los cuestionarios planteando una interfaz nueva que mostrase estadísticas de los mismos, permitiendo al profesorado conocer la actitud y desempeño del alumnado.

Posteriormente se introdujo Sonar Cloud ya que se había añadido bastante código y hacía falta refactorizarlo.

Por temas de tiempo de carga se planteó investigar procesos de asincronía del framework Spring porque al aumentar la funcionalidad de la aplicación también aumentaba el número de llamadas a los web services de Moodle.

Por último, se trabajó en los foros incorporando un conjunto de estadísticas del mismo y se hizo una investigación de las librerías en Java de análisis de sentimiento para analizar los mensajes de cada foro. En este apartado surgieron varios inconvenientes porque la mayoría de librerías solo daban soporte a textos en inglés. Tras varias búsquedas y reuniones se implementaron una serie de llamadas asíncronas a una API que analizaba textos en español y devolvía una nota estimada para dicho texto.

\section{Proceso de obtención de llamadas a los servicios web}
El proceso de obtención de datos de la API de Moodle fue más o menos sencillo ya que había una documentación previa gracias a la primera versión de este proyecto ya que la propia documentación de la página web de Moodle no dejaba muy claro cómo realiza dichas llamadas.
Para la documentación también se eligió utilizar la incluida en la instalación de Moodle ya que es mucho más comprensible.

Para realizar las pruebas de las nuevas llamadas, se utilizó la herramienta Postman. Postman permite enviar solicitudes HTTP a la API y recibir las respuestas correspondientes. Una ventaja significativa de utilizar Postman es que formatea y presenta las respuestas de manera legible, con tabulaciones y estructura clara, a diferencia de la visualización de respuestas en una sola línea que se obtiene al hacer las llamadas directamente desde el navegador. Esto facilita la inspección y comprensión de los datos devueltos por la API durante el proceso de prueba y depuración. Con dichas respuestas en formato JSON (JavaScript Object Notation) se trasladaban a Json2CSharp.com para obtener el código que formaría el nuevo modelo.

Todas estas llamadas se probaron con la página web de pruebas que provee Moodle: Mount Orange School y con un Moodle instalado en local. Este último permitía hacer pruebas más detalladas y comprobar las estadísticas de los cuestionarios que generaba la aplicación con los que mostraba Moodle.\imagen{JSON.png}{Obtención del JSON}

\section{Implementación de GitHub Actions}
En este apartado se incluyó Sonar Cloud en las acciones de GitHub indicando que cada vez que se haga un push en la rama develop se debe superar la \textit{Quality Gate} indicada en Sonar Cloud. Esta \textit{Quality Gate} se puede dejar la que viene por defecto o programar una propia. En este caso se ha implementado una nueva. \imagen{QualityGate.png}{\label{fig:qualitygate}Quality gate usada en el proyecto}
Desde la implementación de SonarCloud en el ciclo de integración y despliegue continuo, se ha establecido el hábito de verificar los errores, defectos y vulnerabilidades generados después de cada commit. Este enfoque previene la acumulación de problemas y, al mismo tiempo, facilita el mantenimiento del código. Asimismo, el proceso de abordar los code smells ha resultado en una disminución de su incidencia.

\section{Generación de estadísticas de los cuestionarios}
Para la generación de estadísticas, Moodle devuelve la información de manera poco eficiente. Para obtener la media de notas de un cuestionario, es necesario realizar una serie de pasos: primero, obtener los cuestionarios del curso; luego, obtener los intentos del cuestionario previamente obtenido; finalmente, calcular las estadísticas necesarias a partir de todos esos intentos, que incluyen tanto la nota global como la nota por pregunta.

Esta forma de obtención de datos resulta poco eficiente, especialmente en casos donde el curso cuenta con un gran número de alumnos. Por ejemplo, en un curso con 50 alumnos que hayan realizado 15 cuestionarios a lo largo del curso, con varios intentos por persona, se requerirían más de 750 llamadas a la API de Moodle solo para generar las estadísticas necesarias.

Esta cantidad de llamadas representa una carga considerable para el sistema, considerando que incluso para clases con un número relativamente bajo de alumnos, resulta una cantidad considerable de solicitudes. Por lo tanto, se hace evidente la necesidad de mejorar el proceso de obtención de estadísticas para optimizar el rendimiento y eficiencia en el manejo de los datos en Moodle.

\section{Cambio de versión de Moodle}
Como Moodle en este último año ha implementado la versión 4.1 algunas llamadas a la API han variado sus respuestas como es el caso de los campos booleanos, antes estos devolvían un valor entero pero con la nueva versión se devuelve un booleano. Esto implicó una reestructuración de algunos modelos que tuviesen campos como isVisible.
\include{./tex/6_Trabajos_relacionados}
\capitulo{7}{Conclusiones y Líneas de trabajo futuras}
\section{Conclusiones}
Con la implementación de esta segunda versión y gracias al trabajo ya realizado previamente, he comprendido la cantidad de carga de trabajo y de necesidad de administración que hay en un curso, por eso he visto muy importante el desarrollo de un proyecto como este. En mi caso he decidido resaltar el área de cuestionarios y foros porque es donde veo una interacción muy cercana con los alumnos. Este tema me ha preocupado desde que llegué a la universidad ya que muchas veces veía una brecha entre el alumno y el profesor, mucho más grande a medida que subían de categoría los estudios y más en la enseñanza a distancia.

En mi caso no es la primera toma de contacto con el e-Learning puesto que cursé el grado superior de Desarrollo de Aplicaciones Multiplataforma (DAM) a distancia y el grado superior de Desarrollo de Aplicaciones Web (DAW) de manera presencial. Por otro lado, cursé primero y segundo de carrera de manera presencial y tercero y cuarto en la modalidad online. Con esto creo que puedo comprender mejor que otras personas las diferencias que hay entre las distintas modalidades y la necesidad de hacer una aplicación que facilite dicha interacción.

Respecto a los objetivos quiero hacer una pequeña recapitulación:
\begin{enumerate}
    \item Definir los modelos y sus respectivos atributos de los cuestionarios, intentos y foros.

    Objetivo ampliamente cumplido ya que todos los datos obtenidos de los web services de Moodle se almacenan en modelos y las estadísticas generadas con los datos recibidos se almacenan en unos modelos especializados para dichos datos.
    
    \item Combinar los servicios Web de Moodle para adaptar los datos obtenidos a las nuevas funcionalidades.

    También conseguido, se han podido obtener todas las estadísticas que queríamos a falta de la aleatoriedad de la nota de la pregunta ya que vimos que era necesario implementar web scraping en la aplicación.
    
    \item Adaptar la información de los cuestionarios y foros a la interfaz ya creada para mantener el aspecto amigable y funcional.

    Creo que este objetivo también se ha conseguido, la interfaz es fácil de entender y hemos dado varias revisiones a la forma de mostrar los datos para que sean intuitivos.
    
    \item Aplicar los frameworks internacionales de calidad en e-learning a los datos que reproducirá el proyecto.

    Otro objetivo alcanzado, todos los cambios introducidos han ido siempre en base a dichos frameworks de calidad.
    
    \item Diseñar indicadores cualitativos y cuantitativos de calidad de cada fase de diseño instruccional del curso en línea (diseño, implementación, realización y evaluación)\cite{previotfg}

    Se ha logrado alcanzar este objetivo al implementar en el plan de calidad indicadores cualitativos para cada una de las verificaciones realizadas. A partir de estos indicadores, se han generado otros que resumen el rendimiento en cada una de las etapas.
\end{enumerate}

Ha sido muy interesante volver a trabajar con JSP como hice hace unos años cuando empecé en la informática y haber salido de mi zona de comfort acostumbrado a trabajar con React, Python y Flask.

Por otro lado, los tutores me han guiado bastante bien porque en algunos puntos no sabía muy bien como cubrir ciertas características, obtener los datos o trabajar con ellos. También han sido capaces de adaptarse a mi disponibilidad y en algunos casos falta de tiempo siendo flexibles a la hora de realizar cambios en la aplicación por lo que estoy agradecido con ellos.

\section{Líneas de trabajo futuras}
En este punto quisiera dejar lo que escribió mi compañero \cite{previotfg} y agregando las posibles mejoras que descubrí a lo largo de estos meses trabajando en esta aplicación.

Aunque la aplicación desarrollada en este TFG es funcionalmente completa existen múltiples factores que hacen que pueda necesitar de futuras adaptaciones. Se considera interesante tener en cuenta factores como: la subjetividad intrínseca asociada a los procesos de calidad y las fuentes tecnológicas para definir controles de calidad automáticos, las diferencias en la distribución de responsabilidades de roles académicos, el tamaño de las instituciones académicas y las diferentes soluciones tecnológicas de implementación de cursos en línea. El análisis detallado de cada factor hace que surjan muchas líneas de trabajo futuras, a continuación se enumeran algunas que se han considerado interesantes.
\begin{itemize}
	\item
	El método que utiliza la aplicación para realizar la comprobación de si el profesor responde a las dudas de los alumnos es una solución preliminar e incompleta. Se tiene como objetivo a futuro encontrar una forma más fiable de determinar qué es una duda y cuándo ha sido resuelta.
	El uso de modelos basados en el procesamiento del lenguaje natural puede ser un campo exploratorio que permita poder clasificar un mensaje del foro como una respuesta de dudas de un profesor. Pensamos que el diseño experimental y el posterior análisis de un clasificador con este cometido es suficientemente complejo para ser considerado un TFG por sí mismo. Otro problema son las situaciones en las que la duda del alumno ha sido respondida por otro alumno y no hace falta responderla o se resuelve la duda en un comentario independiente en el mismo foro y no se detecte como respuesta.
	\item
	En este apartado voy a modificar lo que dijo mi compañero ya que habla de los cuestionarios y esto es algo que se ha implementado con éxito aunque coincido con él en que se puede seguir profundizando en ellos como una optimización de las llamadas o el uso de web scraping para obtener los datos que no ofrece Moodle como la aleatoriedad de la nota previamente mencionada.
	\item
	Por el momento la aplicación solo integra cursos Moodle, pero sería conveniente que la aplicación permitiera analizar cursos online de otros LMS como Blackboard o Edmodo. Sin embargo, realizar los cambios para esto supondría adaptarse a las APIs de servicios correspondientes suponiendo que sean lo suficientemente parecidas, y en caso de no poder acceder a la información necesaria mediante servicios web, habría que implementar otras formas de acceder a la información necesaria.
	\item
	Los plugin de Moodle con los que comparo la aplicación tienen más idiomas disponibles, esto se debe a que están dispuestos de forma que cualquiera pueda aportar sus propias traducciones, sin embargo, el internacionalizar la aplicación la haría más competitiva.
	\item
	La lista de cursos que muestra la aplicación es la lista de los cursos en los que se encuentra matriculado el usuario con independencia de su rol o los permisos que tenga. Sería interesante seguir mostrando los mismos cursos pero sin resaltar en caso de que no se tengan los permisos necesarios para realizar las consultas para poder contactar al administrador en caso de problemas con los permisos.
	\item
	Sería muy interesante poder activar y desactivar las diferentes consultas desde los archivos de configuración si tenemos en cuenta que algunas consultas no aplican para algunos tipos de curso. Sin embargo, a la hora de calcular las estadísticas posteriores habría que adaptarse según las consultas que estén activadas, algo que es difícil teniendo en cuenta que la matriz Rol-Responsabilidad del informe se obtiene multiplicando un vector con los puntos obtenidos de cada consulta por una matriz que contiene la cantidad de puntos que supone el cumplimiento de cada comprobación para cada combinación de rol y perspectiva. El objetivo sería buscar otra forma de realizar el cálculo para facilitar la implementación de lo primero.
	\item
	Implementar una estructura de directorios mejor es algo necesario a largo plazo, la mantenibilidad del proyecto se verá afectada dentro de poco si el número de clases aumenta o el propio controlador se convertirá en una clase imposible de entender por las miles de líneas de código.
	\item
	Obtener la información en tiempo real. Con los puntos de comprobaciones veo importante recibir la información antes de cargar el informe pero hay componentes como las estadísticas de los cuestionarios y foros que podrían pedir la información en tiempo real evitando una saturación en el tiempo de carga. Otra solución sería investigar la asincronía que implementa Spring.
\end{itemize}


%\bibliographystyle{plain}
\bibliographystyle{plainurl}
\bibliography{bibliografia}

\end{document}
