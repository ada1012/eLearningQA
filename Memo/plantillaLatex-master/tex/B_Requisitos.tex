\apendice{Especificación de Requisitos}
\label{apendice:B}
\section{Introducción}
En esta sección se recopilan los requisitos tanto funcionales como no funcionales del software en consideración, basados en los objetivos generales y las expectativas que se han establecido para el proyecto. La especificación de requisitos juega un papel fundamental como canal de comunicación entre todas las entidades involucradas en el desarrollo del software.
\section{Objetivos generales}
A continuación se listarán los objetivos generales del proyecto:
\begin{itemize}
	\item
	Obtener a simple vista la existencia de cuestionarios y foros del curso que se solicita el informe.
	\item
	Poder conocer el porcentaje de participación a nivel global y específico de los cuestionarios y foros del curso.
	\item
	Acceder a una vista más detallada con la información más importante de dichos cuestionarios y foros.
\end{itemize}

\section{Catalogo de requisitos}
Después de hablar de los objetivos generales, se listarán los requisitos del sistema:
\begin{itemize}
	\item R-01: la aplicación mostrará en el apartado de diseño una cruz o un tick por cuestionarios y otro por foros dependiendo de si estos existen en dicho curso.
	\item R-02: la aplicación implementará en la sección de realización una fila desplegable indicando el porcentaje de alumnos que interactua con los cuestionarios. Si se despliega se mostrará una nueva fila por cada cuestionario del curso con el porcentaje de participación de cada uno.
	\item R-03: la aplicación implementará en la sección de realización una fila desplegable indicando el porcentaje de alumnos que interactua con los foros. Si se despliega se mostrará una nueva fila por cada foro del curso con el porcentaje de participación de cada uno.
	\item R-04: al hacer click en alguno de los cuestionarios desplegados se mostrarán una serie de estadísticas junto con un grafo.
	\item R-05: al hacer click en alguno de los foros desplegados se mostrarán una serie de estadísticas.
\end{itemize}