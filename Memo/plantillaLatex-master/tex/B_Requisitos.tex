\apendice{Especificación de Requisitos}
\label{apendice:B}
\section{Introducción}
En esta sección se recopilan los requisitos tanto funcionales como no funcionales del software en consideración, basados en los objetivos generales y las expectativas que se han establecido para el proyecto. La especificación de requisitos juega un papel fundamental como canal de comunicación entre todas las entidades involucradas en el desarrollo del software.
\section{Objetivos generales}
A continuación se listarán los objetivos generales de la nueva versión del proyecto:
\begin{itemize}
	\item
	Obtener a simple vista la existencia de cuestionarios y foros del curso que se solicita el informe.
	\item
	Poder conocer el porcentaje de participación a nivel global y específico de los cuestionarios y foros del curso.
	\item
	Acceder a una vista más detallada con la información más importante de dichos cuestionarios y foros.
\end{itemize}

\section{Catalogo de requisitos}
Después de hablar de los objetivos generales, se listarán los requisitos del sistema:
\begin{itemize}
	\item R-01: la aplicación mostrará en el apartado de diseño una cruz o un tick por cuestionarios y otro por foros dependiendo de si estos existen en dicho curso.
	\item R-02: la aplicación implementará en la sección de realización una fila desplegable indicando el porcentaje de alumnos que interactua con los cuestionarios. Si se despliega se mostrará una nueva fila por cada cuestionario del curso con el porcentaje de participación de cada uno.
	\item R-03: la aplicación implementará en la sección de realización una fila desplegable indicando el porcentaje de alumnos que interactua con los foros. Si se despliega se mostrará una nueva fila por cada foro del curso con el porcentaje de participación de cada uno.
	\item R-04: al hacer click en alguno de los cuestionarios desplegados se mostrarán una serie de estadísticas junto con un grafo.
	\item R-05: al hacer click en alguno de los foros desplegados se mostrarán una serie de estadísticas.
\end{itemize}

\section{Especificación de requisitos}
Para comprender estos requisitos se procederá a crear un diagrama de casos de uso de la versión 2 del proyecto:
\imagen{DiagramaCasosUso.PNG}{Diagrama de casos de uso}

\begin{table}[H]
 	\caption{Caso de uso CU-01}
 	\resizebox{\textwidth}{!}{%
 		\begin{tabular}{cl}
 			\hline
 			\textbf{CU-01}       & \textbf{Generar informe}                                   \\ \hline
 			\textbf{Descripción}          & Implementar los cambios oportunos al informe que se generaba en la versión previa    \\
 			\textbf{Requisitos asociados} & R-01                                                   \\
 			\textbf{Precondiciones}       & Debe acceder con un usuario válido                     \\
 			\textbf{Ejecución}            &\begin{tabular}[l]{@{}l@{}}1.El usuario accede a la aplicación con un usuario, contraseña, dominio y grupo existentes\\ 2.El usuario elige el curso que quiere ver\\ 3.Se mostrarán las 21 comprobaciones existentes del curso.\end{tabular}                                                \\
 			\textbf{Postcondiciones}      &\begin{tabular}[c]{@{}c@{}}Se ha generado o ampliado un archivo csv para ese\\ servidor, profesor y curso\end{tabular}                                                      \\
 			\textbf{Excepciones}          &
 		\end{tabular}%
 	}
 \end{table}

\begin{table}[H]
	\caption{Caso de uso CU-02}
   \resizebox{\textwidth}{!}{%
	   \begin{tabular}{cl}
		   \hline
		   \textbf{CU-05}       & \textbf{Realizar comprobaciones}                                   \\ \hline
		   \textbf{Descripción}          & Se realizan las comprobaciones necesarias para evaluar la calidad de los cursos    \\
		   \textbf{Requisitos asociados} & R-02                                                   \\
		   \textbf{Precondiciones}       & El usuario está logueado en la aplicación y se ha debido generar un token           \\
		   \textbf{Ejecución}            &\begin{tabular}[l]{@{}l@{}}1.El usuario hace click en un curso\\ 2.Se obtiene el token de sesión\\ 3.Se obtiene la información del curso\\ 4.Se calculan los porcentajes de desempeño\\ 5.Se muestran los resultados en la página\end{tabular}                                                \\
		   \textbf{Postcondiciones}      &                                                        \\
		   \textbf{Excepciones}          & 
	   \end{tabular}%
   }
\end{table}

\begin{table}[H]
	\caption{Caso de uso CU-03}
   \resizebox{\textwidth}{!}{%
	   \begin{tabular}{cl}
		   \hline
		   \textbf{CU-05}       & \textbf{Calcular estadísticas cuestionarios}                                   \\ \hline
		   \textbf{Descripción}          & Obtiene los intentos de los cuestionarios y calcula las estadísticas    \\
		   \textbf{Requisitos asociados} & R-03                                                  \\
		   \textbf{Precondiciones}       & El usuario está logueado en la aplicación y se ha debido generar un token           \\
		   \textbf{Ejecución}            &\begin{tabular}[l]{@{}l@{}}1.El usuario accede al informe del curso.\\ 2.Se obtienen los intentos de los cuestionarios.\\ 3.Se calculan las estadísticas de los cuestionarios.\\ 4.Se muestra una comprobación con el porcentaje de participación en los foros del curso.\\ 5.La comprobación tendrá un botón "Desplegar" que desplegará una fila por cada cuestionario del curso.\\ 6.Al hacer click en un cuestionario se mostrarán las estadísticas del cuestionario correspondiente.\end{tabular}                                                \\
		   \textbf{Postcondiciones}      &                                                        \\
		   \textbf{Excepciones}          & 
	   \end{tabular}%
   }
\end{table}

\begin{table}[H]
	\caption{Caso de uso CU-04}
   \resizebox{\textwidth}{!}{%
	   \begin{tabular}{cl}
		   \hline
		   \textbf{CU-05}       & \textbf{Calcular estadísticas foros}                                   \\ \hline
		   \textbf{Descripción}          & Obtiene el porcentaje de participación y mensajes en los foros de un curso    \\
		   \textbf{Requisitos asociados} & R-04                                                  \\
		   \textbf{Precondiciones}       & El usuario está logueado en la aplicación y se ha debido generar un token           \\
		   \textbf{Ejecución}            &\begin{tabular}[l]{@{}l@{}}1.El usuario hace clic en el curso correspondiente\\ 2.Se obtienen los posts de los foros.\\ 3.Se calcula el porcentaje de participación\\ 4.Se muestra una comprobación con el porcentaje de participación en los foros del curso.\\ 5.La comprobación tendrá un botón "Desplegar" que desplegará una fila por cada foro del curso.\\ 6.Al hacer click en un foro se mostrarán las estadísticas del foro correspondiente.\end{tabular}                                                \\
		   \textbf{Postcondiciones}      &                                                        \\
		   \textbf{Excepciones}          & 
	   \end{tabular}%
   }
\end{table}