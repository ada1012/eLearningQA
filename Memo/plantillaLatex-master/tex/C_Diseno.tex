\apendice{Especificación de diseño}

\section{Introducción}
En este apéndice se proporciona una descripción detallada de cómo se tiene previsto cumplir con el conjunto de objetivos y requisitos establecidos en el apéndice \ref{apendice:B}. Se incluye información sobre los datos utilizados por la aplicación, su arquitectura y su diseño de procedimientos.
\section{Diseño de datos}
En la siguiente tabla se recogen las entidades asociadas a las consultas incorporadas en la nueva versión.
\begin{table}[H]
	\resizebox{\textwidth}{!}{%
		\begin{tabular}{|l|l|}
			\hline
			\textbf{Consulta}                                                                                                                                        & \textbf{Entidades}        \\ \hline
			El curso tiene al menos un cuestionario                                                                                                         & Curso, Cuestionario            \\ \hline
			El curso tiene al menos un foro                                                                                                                 & Curso, Foro            \\ \hline
			Un mínimo de alumnos participa en los cuestionarios                                                                                             & Curso, Cuestionario            \\ \hline
			Un mínimo de alumnos participa en los foros                                                                                                     & Curso, Foro            \\ \hline
		\end{tabular}%
	}
\end{table}
Seguidamente se mostrará una imagen del modelo de datos necesario para comprender la estructura con la que trabaja Moodle y los datos que se deberán recuperar.
\imagen{ModeloDatos.png}{Modelo de datos base}
\section{Diseño procedimental}
\imagen{BocetoInterfaz.png}{Boceto de la interfaz}
\subsection{Patrón fachada}
El patrón de diseño Fachada es un patrón estructural que proporciona una interfaz simplificada para interactuar con un sistema complejo o conjunto de clases. La idea principal detrás del patrón Fachada es proporcionar una interfaz de nivel superior que simplifique el uso del sistema subyacente. En lugar de que los clientes interactúen directamente con múltiples componentes o clases internas, pueden hacerlo a través de la fachada, que se encarga de manejar las interacciones internas y ofrecer una interfaz más simple y específica.
\imagen{DiagramaFachada.png}{Diagrama de interacción con la fachada}
A continuación se mostrará la definición de Roberto de la primera versión la cuál no varía y solamente cabría destacar que en este trabajo el patrón de diseño Fachada se ha utilizado para solicitar a la clase WebServiceClient las estadísticas de cuestionarios y foros para luego mostrarlos en el jsp generado.
Seguidamente, se explicará como acceder a la información a través del patrón de diseño fachada:
Cuando un usuario intenta acceder a una página específica, como por ejemplo el informe de la aplicación, la clase SpringController se encarga de manejar la solicitud y proporcionar la respuesta correspondiente. En este caso, la respuesta es un archivo JSP, el cual debe ser compilado y ejecutado para generar una página HTML como resultado final. Al ejecutar el archivo JSP en cuestión, se desencadenan las llamadas necesarias a la fachada para realizar las comprobaciones predefinidas. Estas comprobaciones se llevan a cabo a través de la clase WebServiceClient, que accede al servidor de Moodle. Después de recibir los resultados de las comprobaciones, el archivo JSP utiliza la fachada como una clase de utilidad para procesar algunos de los resultados. Una vez finalizada la ejecución del archivo JSP, la clase SpringController devuelve la página HTML generada al usuario \cite{previotfganexos}.