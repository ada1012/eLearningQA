\capitulo{3}{Conceptos teóricos}


\section{Definiciones básicas}

En primer lugar, quisiera mencionar que en el trabajo previo \cite{previotfg} existe un desglose de palabras clave que permitirá comprender mejor la aplicación, en esta memoria solo se mencionarán las definiciones nuevas.

\begin{itemize}
	\item \textbf{Cuestionario:} es un instrumento de evaluación utilizado por los profesores o facilitadores para medir el conocimiento, comprensión y habilidades adquiridas por los estudiantes durante el proceso de aprendizaje \cite{Lissette}.
	\item \textbf{Asimetría o skewness:} es una medida estadística que describe la asimetría de una distribución de datos. Indica si la distribución de los datos está desplazada hacia la izquierda o hacia la derecha en relación con la distribución normal \cite{skewness}.
	\item \textbf{Coeficiente de Curtosis:} es una medida estadística que describe la forma de una distribución de datos y su grado de apuntamiento en relación con una distribución normal. La curtosis mide la concentración de los datos alrededor de la media y evalúa la presencia de colas o picos en la distribución \cite{curtosis}. 
	\item \textbf{Índice de dificultad:} medida que se utiliza para evaluar la dificultad global del cuestionario o de cada ítem individual. Este índice proporciona información sobre el grado de dificultad que los participantes enfrentan al responder las preguntas del cuestionario \cite{indicededificultad}.
	\item \textbf{Foro:} es una herramienta virtual especialmente diseñada para promover la comunicación y el intercambio de ideas entre los participantes de un curso a distancia, ya sean estudiantes o profesores. Este espacio facilita la discusión y el debate sobre temas específicos relacionados con el curso, permitiendo a los usuarios compartir sus conocimientos, plantear preguntas, brindar respuestas y colaborar de manera colaborativa en el proceso de aprendizaje \cite{CHENG2011253}.
	\item \textbf{Hilo:} se refiere a una unidad de conversación temática dentro del foro. Representa un tema o pregunta específica planteada por un participante del curso, que genera respuestas y comentarios por parte de otros usuarios. Un hilo comienza con el mensaje inicial del participante, que establece el tema de discusión, y a partir de ahí se desarrolla a medida que otros usuarios responden, aportan sus opiniones y generan una conversación en torno al tema. Cada hilo del foro se caracteriza por estar centrado en un asunto particular y permite un seguimiento y organización más efectivos de las discusiones dentro del curso a distancia \cite{CHENG2011253}.  
	\item \textbf{Mensaje:} es una unidad de comunicación escrita realizada por un participante del curso dentro de un hilo de discusión específico. Los mensajes son las respuestas, comentarios o aportes que los usuarios del curso hacen en relación al tema planteado en el hilo de discusión \cite{CHENG2011253}.  
	\item \textbf{Análisis de sentimiento:} el proceso de análisis de sentimiento implica el uso de algoritmos y modelos de procesamiento de lenguaje natural (NLP) para extraer información relevante del texto y determinar la polaridad emocional asociada a cada unidad de texto \cite{sentimentalanalysis}.  
\end{itemize}


\section{Marco de referencia de calidad de MOOQ}
El Marco de Referencia de Calidad de MOOQ (Massive Open Online Quality) es un conjunto de estándares y directrices que se utilizan para evaluar y promover la calidad de los cursos en línea masivos y abiertos (MOOCs). Fue desarrollado por un consorcio de instituciones educativas y organizaciones dedicadas a la educación en línea \cite{stracke2018quality}.

La calidad, en el contexto del e-learning, se refiere a la capacidad de cumplir con las necesidades educativas del alumno. Esto implica garantizar que el material educativo sea de alta calidad y facilite la comprensión y el aprendizaje.

El marco de referencia se centra en diferentes aspectos de la calidad de los MOOCs y proporciona pautas para diseñar, desarrollar, implementar y evaluar estos cursos.

En esta segunda versión los cambios que se han implementado se encuentran en la fase de diseño y realización, pero se mantendrá el resto del información para comprender todo el proyecto. A continuación, se explicarán las fases (esto es un reflejo de la descripción utilizada en la memoria anterior pero imprescindible para comprender el presente trabajo):

\subsection{Fases}
\begin{itemize}
	\item \textbf{Análisis:}
	en esta fase, se realiza un análisis exhaustivo de los requisitos, objetivos y necesidades del curso MOOC. Se investiga el público objetivo, se identifican los objetivos de aprendizaje y se recopilan datos relevantes para informar el diseño del curso.
	\item \textbf{Diseño:}
	en esta etapa, se crea la estructura y el plan detallado del curso MOOC. Se definen los objetivos de aprendizaje, se selecciona y organiza el contenido educativo, y se determinan las estrategias de enseñanza y evaluación. Además, se establecen las interacciones y colaboraciones entre estudiantes e instructores, y se define la apariencia y la navegación de la plataforma del curso. En esta fase es dónde se han incorporado los cambios de la existencia de cuestionarios y foros en la creación del curso.
	\item \textbf{Implementación:}
	en esta fase, se lleva a cabo la creación y producción del contenido del curso MOOC, incluyendo la grabación de videos, el desarrollo de materiales interactivos y la preparación de actividades de aprendizaje. También se configura y personaliza la plataforma del curso, se realizan pruebas técnicas y se prepara la infraestructura necesaria para su lanzamiento.
	\item \textbf{Realización:}
	esta etapa implica la puesta en marcha del curso MOOC y su disponibilidad para los estudiantes. Se gestionan las inscripciones, se inician las interacciones en línea, se brinda apoyo técnico y pedagógico a los participantes, y se fomenta la participación activa de los estudiantes en el curso. En esta fase se han incorporado las estadísticas y participación de cuestionarios y foros.
	\item \textbf{Evaluación:}
	en esta fase final, se realiza una evaluación exhaustiva del curso MOOC para determinar su efectividad y calidad. Se recopilan y analizan datos sobre el desempeño de los estudiantes, se recopilan comentarios y se realizan encuestas de satisfacción. Con base en estos resultados, se realizan ajustes y mejoras al curso para futuras iteraciones.
\end{itemize}

\imagen{CicloFases.png}{Ciclo de fases según el marco de MOOQ \cite{stracke2018quality}}

\subsection{Roles}
En el marco de referencia de calidad de MOOQ, se identifican distintas responsabilidades que son asumidas por una o más personas, conocidas como roles. 
\begin{itemize}
	\item \textbf{Diseñadores:}
	el diseñador es responsable de la planificación y creación del curso MOOC. Su función principal es diseñar la estructura y el contenido del curso, estableciendo los objetivos de aprendizaje, seleccionando los materiales y recursos educativos, y diseñando las actividades de aprendizaje. 
	\item \textbf{Facilitadores:}
	el facilitador es el encargado de guiar y apoyar a los estudiantes durante el desarrollo del curso MOOC. Su rol es fomentar la participación activa de los estudiantes, animar las discusiones y responder a las preguntas y consultas.
	\item \textbf{Proveedores:}
	el proveedor se refiere a la entidad o institución que ofrece el curso MOOC. 
\end{itemize}


\subsection{Perspectivas}
\begin{itemize}
	\item \textbf{Pedagógica:}
	esta perspectiva se centra en los aspectos relacionados con la enseñanza y el aprendizaje. Se enfoca en el diseño instruccional, las estrategias de enseñanza, los objetivos de aprendizaje y la interacción entre estudiantes e instructores. La perspectiva pedagógica busca garantizar que el curso MOOC se base en principios educativos sólidos, promueva la participación activa de los estudiantes, ofrezca oportunidades de aprendizaje significativas y se adapte a las necesidades del público objetivo.
	\item \textbf{Tecnológica:}
	esta perspectiva se refiere a los aspectos tecnológicos y de infraestructura del curso MOOC. Considera la selección y uso de herramientas y plataformas tecnológicas, la accesibilidad, la usabilidad y la calidad técnica del curso. La perspectiva tecnológica busca asegurar que la tecnología utilizada en el MOOC facilite el acceso y la participación de los estudiantes, promueva la interactividad y el uso eficiente de los recursos multimedia, y permita un seguimiento y evaluación adecuados del progreso de los estudiantes.
	\item \textbf{Estratégica:}
	esta perspectiva aborda los aspectos estratégicos y de gestión del curso MOOC. Considera la alineación del MOOC con los objetivos y la visión institucional, la identificación de las necesidades y expectativas de los estudiantes, la planificación de recursos humanos y financieros, la promoción y difusión del curso, así como la evaluación y mejora continua. La perspectiva estratégica busca asegurar que el MOOC se integre de manera efectiva en la estrategia global de la institución y se lleve a cabo de manera sostenible y exitosa.
\end{itemize}

\section{Buenas prácticas de la docencia online}
La docencia en línea, o enseñanza a través de plataformas digitales, requiere la implementación de buenas prácticas para garantizar una experiencia educativa efectiva y de calidad. A continuación, se presentan algunas buenas prácticas de la docencia online, junto con ejemplos que ilustran la aplicación web desarrollada:

\subsection{Diseño instruccional claro y estructurado}
Es importante que el curso en línea tenga una estructura clara y coherente. Esto implica proporcionar una visión general del curso, establecer objetivos de aprendizaje claros, organizar el contenido en módulos o unidades temáticas y ofrecer una guía clara sobre cómo navegar por el curso. 

\subsection{Retroalimentación oportuna y constructiva}
Proporcionar retroalimentación efectiva y oportuna es esencial para el crecimiento y desarrollo de los alumnos. El profesor debe ofrecer comentarios constructivos sobre las tareas y actividades de los estudiantes, resaltar sus fortalezas y ofrecer sugerencias para mejorar.

\subsection{Diferentes formas de interactuar en el curso}
Es muy importante abrir al alumno varias vías para interactuar con un curso específico como pueden ser los foros, tutorías o cuestionarios ya que permiten al alumno sentir cierto dinamismo durante los meses que se encuentre estudiando dicha asignatura o módulo. Esto permite pensar al alumno durante ciertos momentos que se encuentra dentro de un aula dejando de lado la distancia que implica el e-Learning.

\subsection{Conocer la participación}
Conocer la participación en sectores como los foros puede ayudar a comprender si los alumnos se encuentra motivados en la asignatura o en un tema o si por el contrario es necesario buscar alguna forma de fomentar dicha participación.

\section{Plan de calidad para cursos e-learning}
En secciones previas hemos presentado un marco de calidad genérico junto con recomendaciones del Centro de Enseñanza Virtual de la Universidad de Burgos junto a la identificación de unas buenas prácticas de enseñanza en cursos en línea.

A partir de esta documentación en este sección definimos de manera concreta unos indicadores cualitativos y cuantitativos de calidad basados en consultas a entidades de cursos en línea.

Para facilitar la comprensión del conjunto de consultas de calidad se agrupan en las fases del marco de referencia de calidad de MOOQ \cite{stracke2018quality}. La Tabla \ref{table:diseño} muestra el  conjunto  de consultas de  fase de diseño, la Tabla \ref{table:implementación} las de la fase de implementación, la Tabla \ref{table:realización} las de la fase de realización y la Tabla \ref{table:evaluación} las de la fase de evaluación.
En este apartado definiremos qué información intentamos conseguir de cada fase (nos vamos a centrar en las fases de diseño, implementación, realización, y evaluación), rol y perspectiva mediante distintos métodos. La mayoría de consultas provienen de la lista de comprobación del Centro de Enseñanza Virtual de la Universidad de Burgos (UBUCEV) de asignaturas virtuales.
Los procesos indicados en la última columna de las tablas se refieren a los procesos que se indican a partir de la página 10 del documento del marco de referencia de calidad de MOOC \cite{stracke2018quality}. \\
Los cambios introducidos en la versión 2 del proyecto se incluirán en azul para distinguirlos en las tablas. Es importante conocer de un vistazo cuáles son las nuevas funcionalidades o conceptos para ser conscientes del crecimiento del proyecto con sus distintas versiones.\\

\label{consultas}
\begin{table}[H]
\caption{Consultas de diseño \linebreak Leyenda:
	\linebreak Responsabilidad: R=Responsable,X=Involucrado
	\linebreak Perspectivas: P=Pedagógica, T=Tecnológica, E=Estratégica}\label{table:diseño}
\resizebox{\textwidth}{!}{%
	\begin{tabular}{|l|l|l|l|l|l|}
		\hline
		\textbf{Consulta}                                                                                                                             & \textbf{Perspectivas} & \textbf{Diseñador} & \textbf{Facilitador} & \textbf{Proveedor} & \textbf{Proceso} \\ \hline
		\begin{tabular}[c]{@{}l@{}}Las opciones\\ de progreso\\ del estudiante\\ están activadas\end{tabular}                                & P            & R         & X           &           & D-5     \\ \hline
		\begin{tabular}[c]{@{}l@{}}Se proporcionan\\ contenidos en\\ diferentes formatos\end{tabular}                                        & PT           & R         & X           & X         & D-4     \\ \hline
		\begin{tabular}[c]{@{}l@{}}El curso tiene\\ grupos\end{tabular}                                                                      & P            & R         & X           & X         & D-3     \\ \hline
		\begin{tabular}[c]{@{}l@{}}El curso tiene\\ actividades\\ grupales\end{tabular}                                                      & P            & R         & X           & X         & D-3     \\ \hline
		\begin{tabular}[c]{@{}l@{}}\textcolor{blue}{El curso tiene}\\ \textcolor{blue}{cuestionarios}\end{tabular}                                                      & \textcolor{blue}{P}            & \textcolor{blue}{R}         & \textcolor{blue}{X}           & \textcolor{blue}{X}         &      \\ \hline
		\begin{tabular}[c]{@{}l@{}}\textcolor{blue}{El curso tiene}\\ \textcolor{blue}{foros}\end{tabular}                                                      & \textcolor{blue}{P}            & \textcolor{blue}{R}         & \textcolor{blue}{X}           & \textcolor{blue}{X}         &      \\ \hline
		\begin{tabular}[c]{@{}l@{}}Los estudiantes\\ pueden ver las\\ condiciones\\ necesarias para\\ completar una\\ actividad\end{tabular} & P            & R         & X           & X         &      \\ \hline
		\begin{tabular}[c]{@{}l@{}}Todas las\\ actividades tienen\\ la misma nota\\ máxima en el\\ calificador\end{tabular}                  & P            & R         & X           & X         &      \\ \hline
		\begin{tabular}[c]{@{}l@{}}Las preguntas de\\ los cuestionarios\\ tienen\\ retroalimentación\end{tabular}                                        & P           & R         & X           & X         &      \\ \hline
		\begin{tabular}[c]{@{}l@{}}Las preguntas de\\ opción multiple\\ puntúan con una\\ calificación aleatoria\\ estimada de cero\end{tabular}                                        & P           & R         & X           & X         &      \\ \hline
	\end{tabular}%
}
\end{table}

\begin{table}[H]
\caption{Consultas de implementación \linebreak Leyenda:
	\linebreak Responsabilidad: R=Responsable,X=Involucrado
	\linebreak Perspectivas: P=Pedagógica, T=Tecnológica, E=Estratégica}\label{table:implementación}
\resizebox{\textwidth}{!}{%
	\begin{tabular}{|l|l|l|l|l|l|}
		\hline
		\textbf{Consulta}                                                                                                                             & \textbf{Perspectivas} & \textbf{Diseñador} & \textbf{Facilitador} & \textbf{Proveedor} & \textbf{Proceso} \\ \hline
		\begin{tabular}[c]{@{}l@{}}Los recursos\\ están actualizados\end{tabular}                                  & PT           & R         & X           & X         & I-1     \\ \hline
		\begin{tabular}[c]{@{}l@{}}Fechas de apertura\\ y cierre de tareas\\ son correctas\end{tabular}            & PT           & X         & R           & X         & R-2     \\ \hline
		\begin{tabular}[c]{@{}l@{}}Se detallan los\\ criterios de\\ evaluación\\ (rúbricas, ejemplos)\end{tabular} & PT           & R         & X           & X         & R-3     \\ \hline
		\begin{tabular}[c]{@{}l@{}}El calificador no\\ tiene demasiado\\ anidamiento\end{tabular} & PE           & R         & X           & X         &      \\ \hline
		\begin{tabular}[c]{@{}l@{}}Los alumnos\\ están divididos\\ en grupos\end{tabular}                          & TE           & X         &             & R         & I-6     \\ \hline
	\end{tabular}%
}
\end{table}


\begin{table}[H]
\caption{Consultas de realización \linebreak Leyenda:
	\linebreak Responsabilidad: R=Responsable,X=Involucrado
	\linebreak Perspectivas: P=Pedagógica, T=Tecnológica, E=Estratégica}\label{table:realización}
\resizebox{\textwidth}{!}{%
	\begin{tabular}{|l|l|l|l|l|l|}
		\hline
		\textbf{Consulta}                                                                                                                             & \textbf{Perspectivas} & \textbf{Diseñador} & \textbf{Facilitador} & \textbf{Proveedor} & \textbf{Proceso} \\ \hline
		\begin{tabular}[c]{@{}l@{}}El profesor\\ responde en los\\ foros dentro del\\ límite de 48 horas\\ lectivas desde que\\ se plantea la duda\end{tabular} & PT           & X         & R           & X         & R-2     \\ \hline
		\begin{tabular}[c]{@{}l@{}}Se ofrece\\ retroalimentación\\ de las tareas\end{tabular}                                                                   & PT           & X         & R           & X         & R-2     \\ \hline
		\begin{tabular}[c]{@{}l@{}}Las tareas están\\ calificadas\end{tabular}                                                                                  & PT           & X         & R           & X         & R-2     \\ \hline
		\begin{tabular}[c]{@{}l@{}}El calificador\\ muestra cómo\\ ponderan las\\ diferentes tareas\end{tabular}                                                & PT           & X         & R           & X         & R-2     \\ \hline
		\begin{tabular}[c]{@{}l@{}}El tiempo de\\ los cuestionarios\\ está bien ajustado\end{tabular}                                        & P           & R         & X           & X         &      \\ \hline
		\begin{tabular}[c]{@{}l@{}}Los cuestionarios\\ tienen una dificultad\\ estimada entre unos \\valores umbrales\end{tabular}                                        & P           & R         & X           & X         &      \\ \hline
		\begin{tabular}[c]{@{}l@{}}Las preguntas de\\ los cuestionarios\\ son buenas\\ discriminando\cite{discrimination-2021}\end{tabular}                                        & P           & R         & X           & X         &      \\ \hline
		\begin{tabular}[c]{@{}l@{}}\textcolor{blue}{Un porcentaje}\\ \textcolor{blue}{establecido de}\\ \textcolor{blue}{alumnos participa}\\ \textcolor{blue}{en los cuestionarios}\end{tabular}                                                      & \textcolor{blue}{P}            & \textcolor{blue}{X}         & \textcolor{blue}{X}           & \textcolor{blue}{R}         &      \\ \hline
		\begin{tabular}[c]{@{}l@{}}\textcolor{blue}{Un porcentaje}\\ \textcolor{blue}{establecido de}\\ \textcolor{blue}{alumnos participa}\\ \textcolor{blue}{en los foros}\end{tabular}                                                      & \textcolor{blue}{P}            & \textcolor{blue}{X}         & \textcolor{blue}{X}           & \textcolor{blue}{R}         &      \\ \hline
	\end{tabular}%
}
\end{table}

\begin{table}[H]
\caption{Consultas de evaluación \linebreak Leyenda:
	\linebreak Responsabilidad: R=Responsable,X=Involucrado
	\linebreak Perspectivas: P=Pedagógica, T=Tecnológica, E=Estratégica}\label{table:evaluación}
\resizebox{\textwidth}{!}{%
	\begin{tabular}{|l|l|l|l|l|l|}
		\hline
		\textbf{Consulta}                                                                                                                             & \textbf{Perspectivas} & \textbf{Diseñador} & \textbf{Facilitador} & \textbf{Proveedor} & \textbf{Proceso} \\ \hline
		\begin{tabular}[c]{@{}l@{}}La mayoría de\\ alumnos responden\\ a los feedbacks\end{tabular} & PTE          & X         & X           & R         & E-2     \\ \hline
		\begin{tabular}[c]{@{}l@{}}Se utilizan encuestas\\ de opinión\end{tabular}                  & PTE          & X         & X           & R         & E-2     \\ \hline
	\end{tabular}%
}
\end{table}

