\capitulo{2}{Objetivos del proyecto}

El objetivo principal de este trabajo es continuar con el desarrollo de una aplicación web que permita al profesor evaluar las distintas fases de diseño instruccional de un curso de Moodle (diseño, implementación, realización, evaluación), tal como recomiendan algunos frameworks internacionales de calidad en e-learning \cite{previotfg}.

Para cumplir dicho objetivo se ha decidido profundizar en el apartado de los cuestionarios y los foros intentando conseguir un informe detallado en esos áreas. Consiguiendo así una rápida lectura de la viabilidad del curso con la posibilidad de obtener detalles en las posibes zonas de mejora.

A continuación se detallarán los subojetivos que darán pie al cumplimiento del objetivo principal:
\begin{enumerate}
    \item Definir los modelos y sus respectivos atributos de los cuestionarios, intentos y foros.
    \item Combinar los servicios Web de Moodle para adaptar los datos obtenidos a las nuevas funcionalidades.
    \item Adaptar la información de los cuestionarios y foros a la interfaz ya creada para mantener el aspecto amigable y funcional.
    \item Aplicar los frameworks internacionales de calidad en e-learning a los datos que reproducirá el proyecto.
    \item Diseñar indicadores cualitativos y cuantitativos de calidad de cada fase de diseño instruccional del curso en línea (diseño, implementación, realización y evaluación)\cite{previotfg}
\end{enumerate}