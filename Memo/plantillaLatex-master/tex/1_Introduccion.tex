\capitulo{1}{Introducción}

En el presente Trabajo de Fin de Grado (TFG), se aborda el desarrollo de una aplicación innovadora que se centra en el aseguramiento de la calidad en el ámbito del e-learning. El e-learning, o aprendizaje electrónico, ha experimentado un crecimiento significativo en los últimos años. Hoy, la distancia ya no es un obstáculo en la educación; no es necesario asistir a las aulas, sino que desde nuestro lugar podemos acceder a la formación académica a través de la educación virtual. \cite{educacionvirtual2023}

Sin embargo, a medida que el e-learning se ha expandido, también han surgido desafíos en términos de asegurar la calidad de los contenidos y las experiencias de aprendizaje ofrecidas. Es fundamental garantizar que los recursos educativos en línea sean efectivos, accesibles, relevantes y estén diseñados de acuerdo con los estándares y las mejores prácticas establecidas.\cite{buenaspracticas2017}

El objetivo principal de esta aplicación es proporcionar a los usuarios un resumen general de cómo está planteado el curso y la viabilidad del mismo. Esta segunda versión busca un nivel de detalle más profundo en el que se intenta mostrar al usuario un conjunto de estadísticas respecto a los cuestionarios y foros dejando al descubierto el interés del alumnado por dicha asignatura. Dichas estadísticas se podrán interpretar de la forma que el usuario vea conveniente. 

La aplicación permitirá a los usuarios realizar evaluaciones sistemáticas de los cursos en línea, identificando fortalezas y áreas de mejora en cada aspecto relevante. Además, ofrecerá una vista de los principales fallos para mejorar la calidad de los cursos, ayudando a los profesionales del e-learning a desarrollar experiencias de aprendizaje mejor adaptadas al marco de calidad.

A lo largo de este trabajo, se describirá en detalle el proceso de desarrollo de la aplicación, desde el diseño de la arquitectura y la interfaz de usuario, hasta la implementación de las funcionalidades clave. Se abordarán los desafíos técnicos y las decisiones de diseño tomadas, así como las pruebas y validaciones realizadas para garantizar el correcto funcionamiento de la aplicación.

Además, se explorarán y analizarán las normativas, estándares y mejores prácticas relevantes en el ámbito del aseguramiento de la calidad en el e-learning, con el fin de fundamentar y respaldar las funcionalidades proporcionadas por la aplicación.