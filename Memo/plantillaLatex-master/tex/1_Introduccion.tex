\capitulo{1}{Introducción}

En el presente Trabajo de Fin de Grado (TFG), se aborda el desarrollo de una aplicación innovadora que se centra en el aseguramiento de la calidad en el ámbito del e-learning. El e-learning, o aprendizaje electrónico, ha experimentado un crecimiento significativo en los últimos años, convirtiéndose en una alternativa educativa cada vez más popular y relevante en diversos entornos académicos y corporativos.

Sin embargo, a medida que el e-learning se ha expandido, también han surgido desafíos en términos de asegurar la calidad de los contenidos y las experiencias de aprendizaje ofrecidas. Es fundamental garantizar que los recursos educativos en línea sean efectivos, accesibles, relevantes y estén diseñados de acuerdo con los estándares y las mejores prácticas establecidas.

El objetivo principal de esta aplicación es proporcionar a los diseñadores, desarrolladores y responsables de la calidad en el e-learning una herramienta integral para evaluar y mejorar la calidad de los cursos en línea. La aplicación se enfoca en varios aspectos clave del aseguramiento de la calidad, incluyendo el diseño instruccional, la usabilidad, la accesibilidad, la interactividad y la evaluación del aprendizaje. 

La aplicación permitirá a los usuarios realizar evaluaciones sistemáticas de los cursos en línea, identificando fortalezas y áreas de mejora en cada aspecto relevante. Además, ofrecerá pautas y recomendaciones prácticas para mejorar la calidad de los cursos, ayudando a los profesionales del e-learning a desarrollar experiencias de aprendizaje más efectivas y satisfactorias.

A lo largo de este trabajo, se describirá en detalle el proceso de desarrollo de la aplicación, desde el diseño de la arquitectura y la interfaz de usuario, hasta la implementación de las funcionalidades clave. Se abordarán los desafíos técnicos y las decisiones de diseño tomadas, así como las pruebas y validaciones realizadas para garantizar el correcto funcionamiento de la aplicación.

Además, se explorarán y analizarán las normativas, estándares y mejores prácticas relevantes en el ámbito del aseguramiento de la calidad en el e-learning, con el fin de fundamentar y respaldar las funcionalidades y recomendaciones proporcionadas por la aplicación.