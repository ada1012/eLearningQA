\apendice{Plan de Proyecto Software}

\section{Introducción}
Con el fin de abordar el plan del proyecto, desarrollaremos una guía detallada que incluirá los puntos clave y los avances logrados en cada sprint. Además, se registrarán las herramientas seleccionadas, las horas invertidas y otros aspectos relevantes. En cada sprint, se especificarán las fechas de inicio y finalización, los objetivos propuestos y los objetivos cumplidos, junto con los problemas encontrados durante el proceso, lo que nos permitirá mostrar la evolución que hemos logrado hasta el momento.

Además, dentro de la guía del plan de proyecto, abordaremos los requisitos detallados del mismo. Esto incluirá un análisis exhaustivo de las funcionalidades y características que la aplicación debe cumplir para satisfacer las necesidades del cliente y los usuarios finales.

También se contemplará el diseño del proyecto, donde se describirán los aspectos técnicos y estéticos que guiarán el desarrollo.

Además, se proporcionará un manual específico destinado al programador, donde se detallarán las pautas y las mejores prácticas para la codificación de la aplicación. Este manual incluirá información sobre los lenguajes de programación utilizados, las convenciones de nomenclatura, las directrices de estilo de codificación y cualquier otro detalle técnico que facilite el desarrollo del proyecto de manera eficiente y coherente.

Asimismo, se creará un manual orientado al usuario final. Este manual contendrá instrucciones claras y concisas sobre cómo utilizar la aplicación o proyecto una vez finalizado. Se incluirán descripciones detalladas de las funcionalidades, los flujos de trabajo recomendados, las configuraciones disponibles y cualquier otra información relevante para que los usuarios finales puedan aprovechar al máximo la aplicación o proyecto.
\section{Planificación temporal}
En esta sección se estimará el tiempo de trabajo en el TFG tanto en la elaboración de la memoria y anexo como en la del propio programa, detallando los acontecimientos a lo largo del proceso.

La planificación del proyecto se ha llevado a cabo en sprints de una duración de dos semanas. Con una reunión al final de cada sprint para comentar los objetivos del último sprint y los nuevos objetivos, en varias ocasiones fue necesario reducir la duración de dichos sprints a una semana. Para el control de versiones tanto de las memorias como del programa se ha utilizado github con un seguimiento a través de commits dejando un histórico bastante detallado. Además de realizar una release cada vez que se tuviese un prototipo funcional de la aplicación.

A continuación mostraré un resumen de dichos sprints:

\subsection{Sprint 0 (23/02/2023 - 08/03/2023)}
	En esta primera reunión se introdujo el proyecto anterior, posibles herramientas con las que trabajar, recursos necesarios y disponibilidad tanto del alumno como de los tutores. Como objetivos se marcaron: inspeccionar el código ya existente, poner en funcionamiento el anterior proyecto, realizar pequeños cambios en el código, inspeccionar el WebService de moodle, instalación y creación de moodle en localhost para empezar a trabajar en local y crear el esqueleto de los documentos en LaTex.
    También añado a la reunión de este sprint que el profesor Carlos López se encontraba de baja por lo que esta y las posteriores reuniones hasta su incorporación las haría con Raúl Marticorena.

\subsection{Sprint 1 (08/03/2023 - 22/03/2023)}
	Durante el segundo sprint tuve varios problemas tanto para entender cómo funcionaba el código como para hacerlo funcionar ya que Moodle había cambiado a la versión 4.0 con cambios en sus respuesta en las llamadas de los web services. Un ejemplo es el paso de campos como "isVisible" de un entero a un booleano. Dados estos problemas y mi falta de tiempo el profesor Raúl Marticorena y yo indicamos como objetivos seguir con la refactorización, enviar el repositorio de github, probar la aplicación contra mountorange e investigar la integración con Heroku aunque este último punto era el menos prioritario para nosotros.

\subsection{Sprint 2 (22/03/2023 - 31/03/2023)}
	Este sprint se decidió hacer de una semana porque a las dos semanas era semana santa y vimos más prioritario realizar la reunión antes y no aplazar la reunión una semana más dejando el sprinte de tres o cuatro semanas. Durante esta semana conseguí que la aplicación se adaptase completamente a la nueva versión de Moodle. Para el siguiente sprint seguiamos con los mismos objetivos ya que no había avanzado mucho.

\subsection{Sprint 3 (31/03/2023 - 17/04/2023)}
	En la revisión de este sprint mostré la incorporación de los modelos necesarios para poder trabajar con cuestionarios en la aplicación y descubrí que Heroku se había vuelto de pago incluso para las aplicaciones de prueba por lo que dejamos el tema de buscra un servidor online de lado. Se mantienen objetivos para el siguiente sprint y se deja la próxima reunión para dentro de una semana.

\subsection{Sprint 4 (17/04/2023 - 27/04/2023)}
	Para esta reunión se había inorporado el un campo en los informes que mostraba si el curso seleccionado tenía cuestionarios o no. También se añadió un registro que mostraba el porcentaje de alumnos que realizaban los cuestionarios del curso. Esta es la primera vez del trabajo que realizo una release ya que teníamos algo avanzada la aplicación. Para el siguiente sprint decidimos mantener la reunión semanal y pusimos como objetivo realizar un resumen del cuestionario al hacer click en dicho cuestionario junto con algún gráfico.

\subsection{Sprint 5 (27/04/2023 - 03/05/2023)}
	Durante este sprint avancé en la lógica de obtención de datos de dicho resumen pero al llegar la reunión no tenía ninguna implementación definitiva por lo que no se mostraban avances en la interfaz. Para la semana siguiente debería tener implementado ya dicho resumen.

\subsection{Sprint 6 (03/05/2023 - 11/05/2023)}
	Para este sprint ya aparecía un resumen en el cuestionario mostrando datos básicos a falta del gráfico. El profesor Raúl MArticorena vió importante meter alguna estadística como el coeficiente de curtosis (con este valor podríamos saber si hay valores atípicos en las notas) y la asimetría. Para la siguiente semana dejamos como objetivo implementar estos datos y el gráfico.

\subsection{Sprint 7 (11/05/2023 - 18/05/2023)}
	Durante esta semana terminé primero los datos del resumen restantes aunque Raúl Marticorena dijo que se estaban calculando mal por lo que para la siguiente semana tendré que corregir la forma en la que los calcula y respecto al gráfico se implementó la lógica pero el envío de la información de Java a Javascript no tuvo éxito por lo que para la siguiente reunión dejamos como objetivo terminar dicho gráfico, empezar a desarrollar la memoria y comenzar con la parte de foros.

\subsection{Sprint 8 (18/05/2023 - 25/05/2023)}
	En esta reunión se incorporó el profesor Carlos López por lo que los primeros minutos fueron para ver el progreso en el proyecto, después aporto una serie de mejoras y fallos que no habíamos contemplado. Los cambios que se implementaron durante este sprint fueron principalmente arreglar los métodos que realizaban cálculos sobre las estadísticas de los cuestionarios, se consiguió implementar correctamente el gráfico (a falta de normalizar el valor de cada pregunta ya que se llegó a la conclusión de que habría preguntas de diferente puntuación) y se generó una release con el tema de los cuestionarios terminado. La parte de los foros se aplazó al siguiente sprint por falta de tiempo ya que quedaban apenas dos semanas para los exámenes finales.

\section{Estudio de viabilidad}

\subsection{Viabilidad económica}
En este apartado se estiman los costes que tendría la realización del proyecto en un caso real. Como los cambios que se han hecho han sido implementaciones al proyecto ya creado se mantendrá la información del compañero de la primera versión\cite{previotfganexos} y se resaltarán en las tablas los cambios con otro color para comprender qué se ha modificado.

La primera versión del proyecto supuso 8 meses de desarrollo con unas 800 horas de trabajo, esto equivaldría a 5 meses. La segunda versión que detalla el tema de los cuestionarios y foros supone en torno a unas 320 horas que equivale a 2 meses de trabajo.
\begin{table}[H]
	\centering
	\caption{Costes de personal}
	\resizebox{0.5\textwidth}{!}{%
		\begin{tabular}{lc}
			\hline
			\textbf{Concepto}        & \textbf{Coste} \\ \hline
			Sueldo mensual neto      & 1.000,00€      \\
			Retención IRPF (19\%)    & 254,52€        \\
			Seguridad Social (6,35\%) & 85,06€         \\
			Sueldo mensual bruto     & 1.339,58€      \\ \hline
			\textbf{\textcolor{blue}{Total 7 meses}}   & \textcolor{blue}{9.377,06€}      \\ \hline
		\end{tabular}%
	}
\end{table}
Las cuotas a la seguridad social se componen de un 4,70\% de contingencias comunes, un 1,55\% por desempleo de tipo general, y un 0,10\% de formación profesional.

Se supone que el ordenador portátil se amortiza en cinco años y ha sido usado cinco meses.
\begin{table}[H]
	\centering
	\caption{Costes de hardware}
	\resizebox{0.75\textwidth}{!}{%
		\begin{tabular}{lcc}
			\hline
			\textbf{Concepto}        & \textbf{Coste}& \textbf{Coste amortizado} \\ \hline
			Ordenador portátil     & 550€ & 45,83€    \\ \hline
			\textbf{Total}   & 550€ & 45,83€    \\ \hline
		\end{tabular}%
	}
\end{table}

\begin{table}[H]
	\centering
	\caption{Costes varios}
	\resizebox{0.5\textwidth}{!}{%
		\begin{tabular}{lc}
			\hline
			\textbf{Concepto}        & \textbf{Coste} \\ \hline
			Internet      & 150,00€      \\
			Electricidad    & 175,00€        \\ \hline
			\textbf{Total}   & 325,00€      \\ \hline
		\end{tabular}%
	}
\end{table}

A partir de estos costes obtenemos el coste total del proyecto:
\begin{table}[H]
	\centering
	\caption{Costes totales}
	\resizebox{0.5\textwidth}{!}{%
		\begin{tabular}{lc}
			\hline
			\textbf{Concepto}        & \textbf{Coste} \\ \hline
			Personal      & \textcolor{blue}{9.377,06€}      \\
			Hardware    & 45,83€           \\
			Varios     & 325,00€      \\ \hline
			\textbf{\textcolor{blue}{Total 7 meses}}    & \textcolor{blue}{9.747,89€}      \\ \hline
		\end{tabular}%
	}
\end{table}

Para hacer rentable el desarrollo de la aplicación teniendo en cuenta que es de código abierto se podría adoptar un modelo SaaS (Software as a Service) en el que los clientes paguen una suscripción al dueño del software y este a su vez se encargue del hosting y mantenimiento de la aplicación.
Se podrían ofrecer subscripciones a varios niveles ofreciendo distintos niveles de funcionalidad y servicios a distintos precios.\cite{previotfganexos}

\subsection{Viabilidad legal}
\subsubsection{Licencias de software}
Respecto a la viabilidad legal en esta segunda versión no se ha incorporado ninguna librería ni se han utilizado nuevos frameworks por lo que se mantendrá esta sección exactamente igual que en la versión anterior.

Para determinar la licencia software que va a utilizar la aplicación hay que tener en cuenta las licencias utilizadas por las dependencias que utiliza la aplicación.

\begin{table}[H]
	\caption{Licencias del software utilizado}
	\resizebox{\textwidth}{!}{%
		\begin{tabular}{|c|c|c|}
			\hline
			\textbf{Software}   & \textbf{Descripción}                                                                                                         & \textbf{Licencia} \\ \hline
			Spring Framework    & Framework para aplicaciones web                                                                                              & Apache 2.0        \\ \hline
			Tomcat Embed Jasper & \begin{tabular}[c]{@{}c@{}}Implementación de Tomcat que incluye\\ Jasper, el parser de JSP de Tomcat\end{tabular}            & Apache 2.0        \\ \hline
			JUnit               & Framework para tests unitarios en Java                                                                                       & EPL               \\ \hline
			Apache Commons IO   & \begin{tabular}[c]{@{}c@{}}Librería de utilidades varias (usado en \\ traducción de imágenes a arrays de bytes)\end{tabular} & Apache 2.0        \\ \hline
			Apache Log4j        & Librería para registro de logs                                                                                               & Apache 2.0        \\ \hline
			Bootstrap           & \begin{tabular}[c]{@{}c@{}}Librerías CSS y JavaScript para \\ páginas web\end{tabular}                                       & MIT               \\ \hline
			Plotly.js           & \begin{tabular}[c]{@{}c@{}}Librería JavaScript de \\ generación de gráficos\end{tabular}                                     & MIT               \\ \hline
		\end{tabular}%
	}
\end{table}
La licencia pública de Eclipse (EPL) es compatible con estas licencias, a continuación se mencionan sus posibilidades y obligaciones.
\begin{itemize}
	\item\textbf{Permite:} uso, reproducción, distribución, modificación, uso comercial y uso de patentes.
	\item\textbf{Obliga a:} revelar la fuente y el autor, mantener la misma licencia al redistribuir el software, distribuir el software libre de regalías.
	\item\textbf{No permite:} responsabilizar al autor o contribuidores por posibles daños, utilizar marcas propiedad del autor para promoción o publicidad.
\end{itemize}
\subsubsection{Uso del nombre ``Moodle'' en la aplicación}
Debido a que Moodle es una marca registrada y no soy un partner registrado de Moodle, tengo que atenerme a una serie de restricciones con respecto al uso de la palabra ``Moodle'' \cite{moodletrademark-2022}:
\begin{itemize}
	\item No puedo usar logos de ``Moodle'' sin consentimiento escrito de Moodle.
	\item No puedo usar ``Moodle'' en el nombre de mi software, ni en el de mi dominio ni en el de mi empresa.
	\item No puedo usar ``Moodle'' en palabras clave relacionadas con la publicidad.
	\item No puedo usar ``Moodle'' para describir servicios alrededor de Moodle de forma que la gente piense que estoy asociado a Moodle cuando no es así.
\end{itemize}

Al enterarme de esto a mitad del desarrollo he tenido que cambiar a ``eLearningQA'' el nombre del proyecto, el del repositorio, el del despliegue en Heroku para que el dominio deje de contener ``Moodle'', y he aclarado en el readme del proyecto y en la memoria que no estoy asociado con Moodle.\cite{previotfganexos}