\apendice{Plan de Proyecto Software}

\section{Introducción}
Con el fin de abordar el plan del proyecto, desarrollaremos una guía detallada que incluirá los puntos clave y los avances logrados en cada sprint. Además, se registrarán las herramientas seleccionadas, las horas invertidas y otros aspectos relevantes. En cada sprint, se especificarán las fechas de inicio y finalización, los objetivos propuestos y los objetivos cumplidos, junto con los problemas encontrados durante el proceso, lo que nos permitirá mostrar la evolución que hemos logrado hasta el momento.

Además, dentro de la guía del plan de proyecto, abordaremos los requisitos detallados del mismo. Esto incluirá un análisis exhaustivo de las funcionalidades y características que la aplicación debe cumplir para satisfacer las necesidades del cliente y los usuarios finales.

También se contemplará el diseño del proyecto, donde se describirán los aspectos técnicos y estéticos que guiarán el desarrollo.

Además, se proporcionará un manual específico destinado al programador, donde se detallarán las pautas y las mejores prácticas para la codificación de la aplicación. Este manual incluirá la estructura de los directorios identificando cada parte de la aplicación y permitiendo identificar la parte a modificar del c´digo de forma rápida y precisa.

Asimismo, se creará un manual orientado al usuario final. Este manual contendrá instrucciones claras y concisas sobre cómo utilizar la aplicación o proyecto una vez finalizado. Se incluirán descripciones detalladas de las funcionalidades, los flujos de trabajo recomendados, las configuraciones disponibles y cualquier otra información relevante para que los usuarios finales puedan aprovechar al máximo la aplicación o proyecto.
\section{Planificación temporal}
En esta sección se estimará el tiempo de trabajo en el TFG tanto en la elaboración de la memoria y anexo como en la del propio programa, detallando los acontecimientos a lo largo del proceso.

La planificación del proyecto se ha llevado a cabo en sprints de una duración de dos semanas. Con una reunión al final de cada sprint para comentar los objetivos del último sprint y los nuevos objetivos, en varias ocasiones fue necesario reducir la duración de dichos sprints a una semana. Para el control de versiones tanto de las memorias como del programa se ha utilizado github con un seguimiento a través de commits dejando un histórico detallado. Además de realizar una release cada vez que se tuviese un prototipo funcional de la aplicación.

A continuación se mostrará un resumen de dichos sprints:

\subsection{Sprint 0 (23/02/2023 - 08/03/2023)}
	En esta primera reunión se introdujo el proyecto anterior, posibles herramientas con las que trabajar, recursos necesarios y disponibilidad tanto del alumno como de los tutores. Como objetivos se marcaron: inspeccionar el código ya existente, poner en funcionamiento el anterior proyecto, realizar pequeños cambios en el código, inspeccionar el WebService de Moodle, instalación y creación de Moodle en localhost para empezar a trabajar en local y crear el esqueleto de los documentos en LaTex.
    También se comenta en la reunión de este sprint que el profesor Carlos López se encontraba de baja por lo que esta y las posteriores reuniones, hasta su incorporación, las haría con el tutor Raúl Marticorena.

\subsection{Sprint 1 (08/03/2023 - 22/03/2023)}
	Durante el segundo sprint tuve varios problemas, tanto para entender cómo funcionaba el código, como para hacerlo funcionar ya que Moodle había cambiado a la versión 4.0 con cambios en sus respuesta en las llamadas de los web services. Un ejemplo es el paso de campos como \textit{isVisible} de un entero a un booleano. Dados estos problemas y mi falta de tiempo, el profesor Raúl Marticorena y yo indicamos como objetivos seguir con la refactorización, enviar el repositorio de github, probar la aplicación contra mountorange e investigar la integración con Heroku aunque este último punto era el menos prioritario para nosotros.

\subsection{Sprint 2 (22/03/2023 - 31/03/2023)}
	Este sprint se decidió hacer de una semana porque a las dos semanas era semana santa y vimos más prioritario realizar la reunión antes y no aplazar la reunión una semana más dejando el sprint de tres o cuatro semanas. Durante esta semana conseguí que la aplicación se adaptase completamente a la nueva versión de Moodle. Para el siguiente sprint seguiamos con los mismos objetivos ya que no había avanzado mucho.

\subsection{Sprint 3 (31/03/2023 - 17/04/2023)}
	En la revisión de este sprint mostré la incorporación de los modelos necesarios para poder trabajar con cuestionarios en la aplicación y descubrí que Heroku se había vuelto de pago incluso para las aplicaciones de prueba por lo que dejamos el tema de buscar un servidor online de lado. Se mantienen objetivos para el siguiente sprint y se deja la próxima reunión para dentro de una semana.

\subsection{Sprint 4 (17/04/2023 - 27/04/2023)}
	Para esta reunión se había incorporado un campo en los informes que mostraba si el curso seleccionado tenía cuestionarios o no. También se añadió un registro que mostraba el porcentaje de alumnos que realizaban los cuestionarios del curso. Esta es la primera vez del trabajo que se realiza una release ya que teníamos algo avanzada la aplicación. Para el siguiente sprint decidimos mantener la reunión semanal y pusimos como objetivo realizar un resumen del cuestionario al hacer click en dicho cuestionario junto con algún gráfico.

\subsection{Sprint 5 (27/04/2023 - 03/05/2023)}
	Durante este sprint avancé en la lógica de obtención de datos de dicho resumen pero al llegar la reunión no tenía ninguna implementación definitiva por lo que no se mostraban avances en la interfaz. Para la semana siguiente debería tener implementado ya dicho resumen.

\subsection{Sprint 6 (03/05/2023 - 11/05/2023)}
	Para este sprint ya aparecía un resumen en el cuestionario mostrando datos básicos a falta del gráfico. El profesor Raúl Marticorena vió importante meter alguna estadística como el coeficiente de curtosis (con este valor podríamos saber si hay valores atípicos en las notas) y la asimetría. Para la siguiente semana dejamos como objetivo implementar estos datos y el gráfico.

\subsection{Sprint 7 (11/05/2023 - 18/05/2023)}
	Durante la semana, se completaron los datos restantes del resumen. Sin embargo, se detectaron errores en los cálculos según lo mencionado por Raúl Marticorena. Para la próxima semana, se planea corregir la forma en que se realizan dichos cálculos. En cuanto al gráfico, se logró implementar la lógica, pero hubo dificultades en la transferencia de información de Java a Javascript. Como resultado, se establece como objetivo para la próxima reunión finalizar el gráfico, iniciar el desarrollo de la memoria y dar inicio a la sección de foros.

\subsection{Sprint 8 (18/05/2023 - 25/05/2023)}
	En esta reunión se incorporó el profesor Carlos López por lo que los primeros minutos fueron para ver el progreso en el proyecto, después se aportaron una serie de mejoras y fallos que no habíamos contemplado. Los cambios que se implementaron durante este sprint fueron principalmente arreglar los métodos que realizaban cálculos sobre las estadísticas de los cuestionarios, se consiguió implementar correctamente el gráfico (a falta de normalizar el valor de cada pregunta ya que se llegó a la conclusión de que habría preguntas de diferente puntuación) y se generó una release con el tema de los cuestionarios terminado. La parte de los foros se aplazó al siguiente sprint por falta de tiempo ya que quedaban apenas dos semanas para los exámenes finales.

\subsection{Sprint 9 (25/05/2023 - 02/06/2023)}
	Durante este sprint se pulieron algunos fallos detectados por los profesores como la normalización de las notas, el número de intentos y agregar el formato de notas en español (se usan comas en vez de puntos). Luego se agregaron los porcentajes de participación en los foros, se redujo el número de llamadas a la API de Moodle. También se avanzo en la redacción de los anexos. En esta reunión se decidió que si queríamos llegar a tiempo tendría que formalizarse un borrador de la memoria, de los anexos, generar una release totalmente funcional con todos los cambios e implementar los actions de github como es el de ejecutar SonarCloud cada vez que se haga un push a la rama develop.

 \subsection{Sprint 10 (02/06/2023 - 08/06/2023)}
	Esta reunión era clave para ver si se hacía entrega del trabajo en primera convocatoria o en segunda. Al ver que se había desarrollado una memoria y unos anexos que estaban poco elaborados y sin estar revisados, además de comprobar que la aplicación daba algunos fallos, se decidió aplazar a segunda convocatoria. También se decidió poner un poco de organización y aunque tarde pero se empezó a hacer uso de las issues y milestones. De ahora en adelante el objetivo era corregir errores, redactar una memoria de calidad e intentar incluir unas últimas mejoras que harían de este un trabajo más completo.

 \subsection{Sprint 11 (08/06/2023 - 26/06/2023)}
	Durante este sprint, que fue un poco más largo, se corrigieron errores en los cálculos de las estadísticas tanto en los cuestionarios como en los foros, se introdujo una página de carga previa a la muestra del informe ya que el tiempo de carga cada vez era mayor. En la reunión se habló de aconsejar trabajar la asincronía de la aplicación en futuras versiones. Con la aplicación funcionando correctamente comenzó el enfoque en la memoria para el siguiente sprint y en la incorporación de análisis de sentimientos en los foros.

\section{Estudio de viabilidad}

\subsection{Viabilidad económica}
En este apartado se estiman los costes que tendría la realización del proyecto en un caso real. Se tendrán en cuenta costes personales, de hardware y costes varios mostrando dos tablas que detallarán en qué gastos se divide cada uno.

La primera versión del proyecto supuso 8 meses de desarrollo con unas 800 horas de trabajo, esto equivaldría a cinco meses. La segunda versión que detalla el tema de los cuestionarios y foros supone en torno a unas 320 horas que equivale a dos meses de trabajo.
\begin{table}[H]
	\centering
	\caption{Costes de personal}
	\resizebox{0.5\textwidth}{!}{%
		\begin{tabular}{lc}
			\hline
			\textbf{Concepto}        & \textbf{Coste} \\ \hline
			Sueldo mensual neto      & 1.000,00€      \\
			Retención IRPF (19\%)    & 254,52€        \\
			Seguridad Social (6,35\%) & 85,06€         \\
			Sueldo mensual bruto     & 1.339,58€      \\ \hline
			\textbf{Total 2 meses}   & 2.679,16€      \\ \hline
		\end{tabular}%
	}
\end{table}
Las cuotas a la seguridad social se componen de un 4,70\% de contingencias comunes, un 1,55\% por desempleo de tipo general, y un 0,10\% de formación profesional. Como se aprecia en la tabla anterior solo en gastos personales se invertirían 2.679,16€.

Se supone que el ordenador portátil se amortiza en cinco años, en la primera versión se utilizó durante cinco meses y en esta segunda versión se ha utilizado durante dos meses aunque separaremos los gastos de versiones y se mostrarán datos únicamente de esta segunda versión.
\begin{table}[H]
	\centering
	\caption{Costes de hardware}
	\resizebox{0.75\textwidth}{!}{%
		\begin{tabular}{lcc}
			\hline
			\textbf{Concepto}        & \textbf{Coste}& \textbf{Coste amortizado} \\ \hline
			Ordenador portátil     & 550€ & 45,83€    \\ \hline
			\textbf{Total}   & 550€ & 45,83€    \\ \hline
		\end{tabular}%
	}
\end{table}

En la siguiente tabla se desglosarán los costes varios como son el internet y la electricidad, a cada uno de estos le asignamos 30,00€ y 35,00€ respectivamente como coste mensual. Al haber usado coste mensual en la descripción, en la tabla se hará la multiplicación por dos meses mostrando el resultado final.

\begin{table}[H]
	\centering
	\caption{Costes varios}
	\resizebox{0.5\textwidth}{!}{%
		\begin{tabular}{lc}
			\hline
			\textbf{Concepto}        & \textbf{Coste} \\ \hline
			Internet      & 60,00€      \\
			Electricidad    & 70,00€        \\ \hline
			\textbf{Total}   & 130,00€      \\ \hline
		\end{tabular}%
	}
\end{table}

A partir de estos costes obtenemos el coste total del proyecto:
\begin{table}[H]
	\centering
	\caption{Costes totales}
	\resizebox{0.5\textwidth}{!}{%
		\begin{tabular}{lc}
			\hline
			\textbf{Concepto}        & \textbf{Coste} \\ \hline
			Personal      & 2.679,16€      \\
			Hardware    & 45,83€           \\
			Varios     & 130,00€      \\ \hline
			\textbf{Total 2 meses}    & 2.854,99€      \\ \hline
		\end{tabular}%
	}
\end{table}

Para rentabilizar la aplicación web se podrían tener en cuenta varias posibles soluciones como ofrecer un sistema freenium dejando parte del contenido de forma gratuita mientras que otras otros apartados serían premium ocasionando un coste como es el caso de Spotify. Otra posible solución es la que comenta mi compañero Roberto Arasti en la versión previa mencionando una suscripción mensual los que implicaría unos ingresos continuos \cite{previotfganexos}.

\subsection{Viabilidad legal}
\subsubsection{Licencias de software}
Respecto a la viabilidad legal en esta segunda versión, solo se ha incorporado GitHub Copilot el cual no ofrece ningún problema legal ya que sus suscripción mensual permite el libre manejo del código generado. A continuación se mostrarán los frameworks y librerías con sus respectivas licencias.

\begin{table}[H]
	\caption{Licencias del software utilizado}
	\resizebox{\textwidth}{!}{%
		\begin{tabular}{|c|c|c|}
			\hline
			\textbf{Software}   & \textbf{Descripción}                                                                                                         & \textbf{Licencia} \\ \hline
			Spring Framework    & Framework para aplicaciones web                                                                                              & Apache 2.0        \\ \hline
			Tomcat Embed Jasper & \begin{tabular}[c]{@{}c@{}}Implementación de Tomcat que incluye\\ Jasper, el parser de JSP de Tomcat\end{tabular}            & Apache 2.0        \\ \hline
			JUnit               & Framework para tests unitarios en Java                                                                                       & EPL               \\ \hline
			Apache Commons IO   & \begin{tabular}[c]{@{}c@{}}Librería de utilidades varias (usado en \\ traducción de imágenes a arrays de bytes)\end{tabular} & Apache 2.0        \\ \hline
			Apache Log4j        & Librería para registro de logs                                                                                               & Apache 2.0        \\ \hline
			Bootstrap           & \begin{tabular}[c]{@{}c@{}}Librerías CSS y JavaScript para \\ páginas web\end{tabular}                                       & MIT               \\ \hline
			Plotly.js           & \begin{tabular}[c]{@{}c@{}}Librería JavaScript de \\ generación de gráficos\end{tabular}                                     & MIT               \\ \hline
		\end{tabular}%
	}
\end{table}

La licencia pública de Eclipse (EPL) es compatible con estas licencias, a continuación se mencionan sus posibilidades y obligaciones.
\begin{itemize}
	\item\textbf{Permite:} uso, reproducción, distribución, modificación, uso comercial y uso de patentes.
	\item\textbf{Obliga a:} revelar la fuente y el autor, mantener la misma licencia al redistribuir el software, distribuir el software libre de regalías.
	\item\textbf{No permite:} responsabilizar al autor o contribuidores por posibles daños, utilizar marcas propiedad del autor para promoción o publicidad.
\end{itemize}

GitHub Copilot es una herramienta de programación desarrollada por GitHub y OpenAI. Utiliza la tecnología de inteligencia artificial de GPT-3 para ofrecer sugerencias de código en tiempo real mientras escribes. Copilot analiza el contexto del código que estás escribiendo y genera automáticamente fragmentos de código relevantes y útiles para completar tus líneas de código.
Tiene dos tarifas:
\begin{itemize}
	\item 10 dólares al mes para uso personal.
	\item 19 dólares al mes por persona para uso empresarial.
\end{itemize}