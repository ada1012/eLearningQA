\apendice{Documentación de usuario}

\section{Introducción}
En este apéndice se presentan los aspectos relevantes de la aplicación que son de interés para el usuario, tales como los requisitos necesarios para utilizar la aplicación y un manual de usuario. Además, se destaca que el manual de usuario está disponible en formato web y puede ser accedido desde la página de inicio de sesión de la aplicación.
\section{Requisitos de usuarios}
Los requisitos como usuario son:
\begin{itemize}
	\item Una cuenta en un servidor de Moodle con rol de profesor en los cursos a analizar
	\item Conexión a internet para acceder al despliegue de la aplicación y para poder obtener información de dichos cursos.
\end{itemize}
\section{Instalación}
Como se ha mencionado en temas anteriores con Java ya instalado el único requisito es ejecutar el archivo WAR que viene en el código fuente de la aplicación.
\section{Manual del usuario}

\subsubsection{Pantallas:}

\textbf{Página de login:} la página de inicio de sesión de la aplicación incluye un formulario donde los usuarios pueden ingresar sus credenciales para acceder al sistema. Además, se encuentran disponibles enlaces adicionales para facilitar la navegación y brindar información adicional. Estos enlaces incluyen un enlace ``Acerca de'' que proporciona detalles sobre la aplicación, un enlace ``Manual de usuario' que redirige a la versión web del manual de usuario, y un enlace de contacto para comunicarse con el equipo de soporte.
\imagen{Login.png}{Página de login}

\textbf{Página principal:} la página principal de la aplicación presenta una cabecera que muestra el nombre de usuario registrado, ofreciendo así una experiencia personalizada. Además, incluye un botón para desconectarse y finalizar la sesión del usuario. También se encuentra disponible un enlace que permite generar un informe global que abarca todos los cursos.

En la página principal, se muestra una tabla que contiene una lista de enlaces para generar informes específicos de cada curso. Estos enlaces permiten acceder rápidamente a los informes detallados de cada curso en particular. Además, se incluye una caja de búsqueda que facilita la búsqueda y filtrado de cursos dentro de la tabla. Esto proporciona una manera conveniente de acceder a la información deseada de manera eficiente.
\imagen{ListaCursos.png}{Página principal}

\textbf{Página de informe:} la página de informe variará dependiendo de si se ha hecho uso de la opción ``Generar informe global'' o por el contrario si se ha elegido la opción de un curso en específico.~

Los informes están compuestos por los siguientes apartados:

\begin{itemize}
	\item
	\textbf{Puntuaciones por rol y perspectiva:} el informe incluye una matriz que evalúa y asigna puntuaciones a cada uno de los roles (Diseñador, Facilitador y Proveedor) en relación a las tres perspectivas establecidas (Pedagógica, Tecnológica y Estratégica). Esta matriz refleja los resultados del análisis y proporciona una visión comparativa de cómo se desempeñan los roles en cada una de las perspectivas mencionadas. A través de las puntuaciones asignadas, se puede evaluar el grado de cumplimiento y efectividad de cada rol en relación a las diferentes dimensiones pedagógicas, tecnológicas y estratégicas identificadas en el informe.
	\item
	\textbf{Gráfico de evolución:} la aplicación guarda las puntuaciones cada vez que se genera un informe, permitiendo así la creación de un gráfico interactivo que muestra la evolución de estos valores a lo largo del tiempo. 
	\imagen{Evolucion.png}{Página de evolución del rendimiento}
	\item
	\textbf{Informe de fases:} en el informe, se incluye una tabla que presenta los resultados de cada una de las comprobaciones realizadas en el análisis. Estas comprobaciones están agrupadas en las diferentes fases del diseño instruccional, que incluyen Diseño, Implementación, Realización y Evaluación. Además, se proporciona una puntuación general que resume el desempeño global del proyecto.

    En el caso específico de los cuestionarios y foros, se presenta una lista desplegable que permite desglosar los resultados de cada uno de ellos. En esta lista, se muestra el porcentaje de participación de cada cuestionario o foro, así como el porcentaje de participación general. Esta información brinda una visión detallada de la contribución de cada componente y su impacto en el conjunto del proyecto, permitiendo una evaluación más precisa y completa.
	\item
	\textbf{Lista de mejoras:} en el informe, se presenta una lista de motivos concretos que han generado resultados negativos en el análisis. Estos motivos son identificados para poder abordarlos y tenerlos en cuenta en futuras acciones. Al hacer clic en las filas de la tabla, se tiene la capacidad de filtrar la lista de mejoras, lo que permite una exploración más específica y focalizada de los aspectos que requieren atención y solución.
	\item
	\textbf{Informe de estadísticas:} finalmente, el informe presenta una sección dedicada a las estadísticas de los cuestionarios y foros. En el caso de los cuestionarios, se muestra un gráfico que representa el índice de dificultad de las preguntas, lo que proporciona una visión clara sobre la complejidad del contenido evaluado. En cuanto a los foros, se incluye una lista que muestra los distintos hilos de discusión presentes en ellos, brindando una visión general de los temas abordados y fomentando la participación activa de los usuarios en las conversaciones relevantes. Además, en cada foro se muestra un análisis de sentimiento de los mensajes que indicarán si tienen una connotación positiva o negativa.
	\imagen{InformeFases.png}{Página de informe de fases}
	\imagen{EstadisticasCuestionario.png}{Estadísticas de un cuestionario}
	\imagen{EstadisticasForo.png}{Estadísticas de un foro}
\end{itemize}

En el informe específico, los resultados de las comprobaciones en el informe de fases se presentan de manera absoluta, es decir, se indican simplemente si el criterio se cumple o no. Sin embargo, en el informe global, se utiliza un enfoque diferente. En lugar de mostrar resultados individuales para cada curso, se evalúa la cantidad de cursos que cumplen cada criterio. De esta manera, se muestra un resultado diferente según el número de cursos que satisfacen dicho criterio.


\subsubsection{Acciones del usuario:}

\begin{itemize}
	\item
	\textbf{Login:} para acceder a la aplicación son necesarias las
	credenciales de acceso a una cuenta de la plataforma Moodle a la que
	accede la aplicación (en el caso del prototipo es Mount ~Orange
	School). Debe introducir su usuario y contraseña en los campos
	``Username'' y ``Password''. Si quiere vaciar los campos pulse el botón
	``Borrar''. Para acceder a la página principal, pulse el botón ``Entrar''
	tras haber introducido sus credenciales.
	\item
	\textbf{Desconectar:} desde la página principal, si desea finalizar su
	sesión, pulse el botón ``Desconectar''. Esto invalidará sus credenciales
	y le impedirá acceder a la aplicación hasta que se registre de nuevo
	con unas credenciales válidas.
	\item
	\textbf{Generar informe específico:} la página principal muestra una
	tabla con todos los cursos en los que se encuentra matriculado el
	usuario registrado en formato de enlace. Al clicar un enlace, se
	generará un informe en una pestaña aparte del navegador que mostrará
	los resultados del análisis que ha realizado la aplicación sobre el
	curso correspondiente.
	\item
	\textbf{Generar informe global:} en la página principal hay un enlace
	llamado ``Generar informe global''. Al hacer clic sobre este, se
	generará un informe en una pestaña aparte del navegador que mostrará
	un resumen de los análisis de todos los cursos en los que se encuentra
	matriculado el profesor.
\end{itemize}

\subsubsection{Explicación de las comprobaciones de los informes:}

Las siguientes comprobaciones están relacionadas con los roles, fases, y
perspectivas mencionados anteriormente. Los distintos procesos del
diseño instruccional se encuentran divididos en fases, ~con ciertas
perspectivas en mente, y son responsabilidad directa o indirecta de
ciertos roles. Al estar las comprobaciones ligadas a esos procesos se
muestran agrupadas por fases, y después de la explicación se indican los
roles responsables e involucrados, además de las perspectivas
correspondientes.

\textbf{Diseño:}

\begin{itemize}
	\item
	\textbf{Las opciones de progreso del estudiante están activadas}: se
	comprueba que estén habilitadas las opciones de progreso de los
	estudiantes en el curso.~{Responsable:} Diseñador
	{Involucrados:} Facilitador {Perspectivas:} Pedagógica
	\item
	\textbf{Se proporcionan contenidos en diferentes formatos:} se
	comprueba que haya variedad de formatos en los recursos del curso.
	{Responsable:} Diseñador {Involucrados:} Facilitador y
	Proveedor {Perspectivas:} Pedagógica y Tecnológica
	\item
	\textbf{El curso tiene grupos:} se comprueba que existan grupos
	definidos en el curso. {Responsable:} Diseñador
	{Involucrados:} Facilitador y Proveedor {Perspectivas:}
	Pedagógica
	\item
	\textbf{El curso tiene actividades grupales:} se comprueba que existan
	actividades con entrega grupal habilitada en el curso.
	{Responsable:} Diseñador {Involucrados:} Facilitador y
	Proveedor {Perspectivas:} Pedagógica
	\item
	\textbf{Los estudiantes pueden ver las condiciones necesarias para
		completar una actividad:} se comprueba que esté habilitada la opción
	de mostrar las condiciones para completar una actividad en el curso.
	{Responsable:} Diseñador {Involucrados:} Facilitador y
	Proveedor {Perspectivas:} Pedagógica
	\item
	\textbf{Todas las actividades tienen la misma nota máxima en el
		calificador:} se comprueba que exista una consistencia en las notas
	máximas de los items de calificación (tareas, entregas, cuestionarios)
	del curso. {Responsable:} Diseñador {Involucrados:}
	Facilitador y Proveedor {Perspectivas:} Pedagógica
	\item
	\textbf{El curso tiene al menos un cuestionario:} se comprueba que exista un cuestionario dentro de la planificación del curso.
    {Responsable:} Diseñador {Involucrados:} Facilitador y Proveedor {Perspectivas:} Pedagógica
	\item
	\textbf{El curso tiene al menos un foro:} se comprueba que exista un foro dentro de la planificación del curso.
    {Responsable:} Diseñador {Involucrados:} Facilitador y Proveedor {Perspectivas:} Pedagógica
\end{itemize}

\textbf{Implementación:}

\begin{itemize}
	\item
	\textbf{Los recursos están actualizados:} se comprueba que los
	recursos del curso tengan una fecha de creación reciente.
	{Responsable:} Diseñador {Involucrados:} Facilitador y
	Proveedor {Perspectivas:} Pedagógica y Tecnológica
	\item
	\textbf{Fechas de apertura y cierre de tareas son correctas:} se
	comprueba que las fechas de apertura y cierre de tareas y
	cuestionarios no se solapen de forma erronea con las fechas de inicio
	y fin del curso. {Responsable:} Facilitador
	{Involucrados:} Diseñador y Proveedor {Perspectivas:}
	Pedagógica y Tecnológica
	\item
	\textbf{Se detallan los criterios de evaluación:} se comprueba que
	exista en al menos una actividad una rúbrica o una guía de calificación
	en el curso.~{Responsable:} Diseñador {Involucrados:}
	Facilitador y Proveedor {Perspectivas:} Pedagógica y Tecnológica
	\item
	\textbf{El calificador no tiene demasiado anidamiento:} se comprueba
	que la estructura de las categorías del calificador no sea demasiado
	enrevesada. {Responsable:} Diseñador {Involucrados:}
	Facilitador y Proveedor {Perspectivas:} Pedagógica y Estratégica
	\item
	\textbf{Todos los alumnos están en algún grupo:} se comprueba que cada
	alumno pertenezca a un grupo. {Responsable:} Proveedor
	{Involucrados:} Diseñador {Perspectivas:} Tecnológica y
	Estratégica
\end{itemize}

\textbf{Realización:}

\begin{itemize}
	\item
	\textbf{El profesor responde en los foros dentro del límite de 48
		horas lectivas desde que se plantea la duda:} se comprueba que no
	hayan preguntas por parte de alumnos que estén sin responder en un
	tiempo razonable. {Responsable:} Facilitador
	{Involucrados:} Diseñador y Proveedor {Perspectivas:}
	Pedagógica y Tecnológica
	\item
	\textbf{Se ofrece retroalimentación de las tareas:} se comprueba que
	el profesor deje comentarios en la mayoría de calificaciones que haga.
	{Responsable:} Facilitador {Involucrados:} Diseñador y
	Proveedor {Perspectivas:} Pedagógica y Tecnológica
	\item
	\textbf{Las tareas están calificadas:} se comprueba que no hayan
	entregas de alumnos que hayan pasado una semana sin calificación.
	{Responsable:} Facilitador {Involucrados:} Diseñador y
	Proveedor {Perspectivas:} Pedagógica y Tecnológica
	\item
	\textbf{El calificador muestra cómo ponderan las diferentes tareas:}
	se comprueba que el calificador muestre los pesos de los items de
	calificación. {Responsable:} Facilitador {Involucrados:}
	Diseñador y Proveedor {Perspectivas:} Pedagógica y Tecnológica
	\item
	\textbf{Al menos un X \% de los alumnos responden a los cuesionarios:}
	se comprueba que hay un mínimo de alumnos que interactúan con los cuestionarios propuestos para el desarrollo del curso. {Responsable:} Proveedor {Involucrados:}
	Diseñador y Facilitador {Perspectivas:} Pedagógica
	\item
	\textbf{Al menos un X \% de los alumnos participa en los foros:}
	se comprueba que hay un mínimo de alumnos que interactúan con los foros propuestos para el desarrollo del curso. {Responsable:} Proveedor {Involucrados:}
	Diseñador y Facilitador {Perspectivas:} Pedagógica
\end{itemize}

\textbf{Evaluación:}

\begin{itemize}
	\item
	\textbf{La mayoría de alumnos responden a los feedbacks:} se comprueba
	que no hayan muchos alumnos que no respondan a los feedbacks.
	{Responsable:} Proveedor {Involucrados:} Diseñador y
	Facilitador {Perspectivas:} Pedagógica, Tecnológica, y
	Estratégica
	\item
	\textbf{Se utilizan encuestas de opinión:} se comprueba que el curso
	contenga encuestas de opinión. {Responsable:} Proveedor
	{Involucrados:} Diseñador y Facilitador {Perspectivas:}
	Pedagógica, Tecnológica, y Estratégica
\end{itemize}

