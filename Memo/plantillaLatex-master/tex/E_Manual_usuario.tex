\apendice{Documentación de usuario}

\section{Introducción}
En este apéndice se incluyen los detalles de la aplicación que conciernen al usuario, como los requisitos para utilizar la aplicación y un manual de usuario que también existe en formato web y puede ser accedido desde la página de login de la aplicación.
\section{Requisitos de usuarios}
Para utilizar la aplicación como usuario se necesita:
\begin{itemize}
	\item Una cuenta en un servidor de Moodle con rol de profesor en los cursos a analizar
	\item Conexión a internet para acceder al despliegue de la aplicación
\end{itemize}
\section{Instalación}
Como se ha mencionado en temas anteriores con Java ya instalado el único requisito es ejecutar el archivo WAR que viene en el código fuente de la aplicación.
\section{Manual del usuario}

\subsubsection{Pantallas:}

\textbf{Página de login:} la página de login contiene un formulario de
inicio de sesión, un link ``Acerca de'', un link ``Manual de usuario'' que
muestra la versión web de este manual, y un link de contacto.
\imagen{Login.png}{Página de login}

\textbf{Página principal:~}la página principal contiene una cabecera que
indica el nombre de usuario con el que se ha registrado, un botón para
desconectarse, un enlace para generar un informe global de todos los
cursos, y una tabla con una lista de enlaces para generar los informes
específicos de cada curso con una caja de búsqueda.
\imagen{ListaCursos.png}{Página principal}

\textbf{Página de informe:} la página de informe genera el informe
dependiendo de cómo se haya accedido a esta, ya sea mediante el enlace
``Generar informe global'' o por medio de los enlaces a informes de cursos
específicos.~

Los informes contienen los siguientes elementos:

\begin{itemize}
	\item
	\textbf{Puntuaciones por rol y perspectiva:} el informe contiene una
	matriz de roles y perspectivas que da distintas puntuaciones a cada
	uno de los roles (Diseñador, Facilitador, y Proveedor) en cada una de
	las tres perspectivas definidas (Pedagógica, Tecnológica, y
	Estratégica) según los resultados del análisis.
	\item
	\textbf{Gráfico de evolución:} la aplicación almacena estas puntuaciones cada vez que se carga un informe para generar un gráfico interactivo de la evolución de estos valores.
	\imagen{Evolucion.png}{Página de evolución del rendimiento}
	\item
	\textbf{Informe de fases:} el informe contiene también una tabla que
	muestra los resultados en cada una de las comprobaciones del análisis y
	las agrupa en las distintas fases del diseño instruccional (Diseño,
	Implementación, Realización, y Evaluación) además de dar una
	puntuación general. En el caso de los cuestionarios y foros tendrán cada uno una lista desplegable en el que se podrá ver un desglose de los mismos con el porcentaje de participación de cada uno y el porcenaje de participación general.
	\item
	\textbf{Lista de mejoras:} el informe muestra una lista de
	motivos concretos que producen resultados negativos en el análisis
	para subsanarlos o tenerlos en cuenta de cara al futuro. Haciendo clic en las filas de la tabla se puede filtrar la lista de mejoras.
	\item
	\textbf{Informe de estadísticas:} por último, el informe muestra una sección de estadísticas de los cuestionarios y foros, en el caso de los cuestionarios se mostrará además un gráfico con el índice de dificultad de las preguntas y en el de los foros una lista con los hilos del mismo.
	\imagen{InformeFases.png}{Página de informe de fases}
	\imagen{EstadisticasCuestionario.png}{Estadísticas de un cuestionario}
	\imagen{EstadisticasForo.png}{Estadísticas de un foro}
\end{itemize}

El informe especifico muestra los resultados de las comprobaciones en el informe de fases de forma absoluta, es decir, que solo indica si el criterio se cumple o no, sin embargo, en el informe global, se muestra un resultado u otro ~dependiendo de la cantidad de cursos que satisfacen dicho criterio.


\subsubsection{Acciones del usuario:}

\begin{itemize}
	\item
	\textbf{Login:} para acceder a la aplicación son necesarias las
	credenciales de acceso a una cuenta de la plataforma Moodle a la que
	accede la aplicación (en el caso del prototipo es Mount ~Orange
	School). Debe introducir su usuario y contraseña en los campos
	``Username'' y ``Password''. Si quiere vaciar los campos pulse el botón
	``Borrar''. Para acceder a la página principal, pulse el botón ``Entrar''
	tras haber introducido sus credenciales.
	\item
	\textbf{Desconectar:} desde la página principal, si desea finalizar su
	sesión, pulse el botón ``Desconectar''. Esto invalidará sus credenciales
	y le impedirá acceder a la aplicación hasta que se registre de nuevo
	con unas credenciales válidas.
	\item
	\textbf{Generar informe específico:} la página principal muestra una
	tabla con todos los cursos en los que se encuentra matriculado el
	usuario registrado en formato de enlace. Al clicar un enlace, se
	generará un informe en una pestaña aparte del navegador que mostrará
	los resultados del análisis que ha realizado la aplicación sobre el
	curso correspondiente.
	\item
	\textbf{Generar informe global:} en la página principal hay un enlace
	llamado ``Generar informe global''. Al hacer clic sobre este, se
	generará un informe en una pestaña aparte del navegador que mostrará
	un resumen de los análisis de todos los cursos en los que se encuentra
	matriculado el profesor.
\end{itemize}

\subsubsection{Explicación de las comprobaciones de los informes:}

Las siguientes comprobaciones están relacionadas con los roles, fases, y
perspectivas mencionados anteriormente. Los distintos procesos del
diseño instruccional se encuentran divididos en fases, ~con ciertas
perspectivas en mente, y son responsabilidad directa o indirecta de
ciertos roles. Al estar las comprobaciones ligadas a esos procesos se
muestran agrupadas por fases, y después de la explicación se indican los
roles responsables e involucrados, además de las perspectivas
correspondientes.

\textbf{Diseño:}

\begin{itemize}
	\item
	\textbf{Las opciones de progreso del estudiante están activadas}: se
	comprueba que estén habilitadas las opciones de progreso de los
	estudiantes en el curso.~{Responsable:} Diseñador
	{Involucrados:} Facilitador {Perspectivas:} Pedagógica
	\item
	\textbf{Se proporcionan contenidos en diferentes formatos:} se
	comprueba que haya variedad de formatos en los recursos del curso.
	{Responsable:} Diseñador {Involucrados:} Facilitador y
	Proveedor {Perspectivas:} Pedagógica y Tecnológica
	\item
	\textbf{El curso tiene grupos:} se comprueba que existan grupos
	definidos en el curso. {Responsable:} Diseñador
	{Involucrados:} Facilitador y Proveedor {Perspectivas:}
	Pedagógica
	\item
	\textbf{El curso tiene actividades grupales:} se comprueba que existan
	actividades con entrega grupal habilitada en el curso.
	{Responsable:} Diseñador {Involucrados:} Facilitador y
	Proveedor {Perspectivas:} Pedagógica
	\item
	\textbf{Los estudiantes pueden ver las condiciones necesarias para
		completar una actividad:} se comprueba que esté habilitada la opción
	de mostrar las condiciones para completar una actividad en el curso.
	{Responsable:} Diseñador {Involucrados:} Facilitador y
	Proveedor {Perspectivas:} Pedagógica
	\item
	\textbf{Todas las actividades tienen la misma nota máxima en el
		calificador:} se comprueba que exista una consistencia en las notas
	máximas de los items de calificación (tareas, entregas, cuestionarios)
	del curso. {Responsable:} Diseñador {Involucrados:}
	Facilitador y Proveedor {Perspectivas:} Pedagógica
	\textbf{El curso tiene al menos un cuestionario:} se comprueba que exista un cuestionario dentro de la planificación del curso.
    {Responsable:} Diseñador {Involucrados:} Facilitador y Proveedor {Perspectivas:} Pedagógica
	\textbf{El curso tiene al menos un foro:} se comprueba que exista un foro dentro de la planificación del curso.
    {Responsable:} Diseñador {Involucrados:} Facilitador y Proveedor {Perspectivas:} Pedagógica
\end{itemize}

\textbf{Implementación:}

\begin{itemize}
	\item
	\textbf{Los recursos están actualizados:} se comprueba que los
	recursos del curso tengan una fecha de creación reciente.
	{Responsable:} Diseñador {Involucrados:} Facilitador y
	Proveedor {Perspectivas:} Pedagógica y Tecnológica
	\item
	\textbf{Fechas de apertura y cierre de tareas son correctas:} se
	comprueba que las fechas de apertura y cierre de tareas y
	cuestionarios no se solapen de forma erronea con las fechas de inicio
	y fin del curso. {Responsable:} Facilitador
	{Involucrados:} Diseñador y Proveedor {Perspectivas:}
	Pedagógica y Tecnológica
	\item
	\textbf{Se detallan los criterios de evaluación:} se comprueba que
	exista en al menos una actividad una rúbrica o una guía de calificación
	en el curso.~{Responsable:} Diseñador {Involucrados:}
	Facilitador y Proveedor {Perspectivas:} Pedagógica y Tecnológica
	\item
	\textbf{El calificador no tiene demasiado anidamiento:} se comprueba
	que la estructura de las categorías del calificador no sea demasiado
	enrevesada. {Responsable:} Diseñador {Involucrados:}
	Facilitador y Proveedor {Perspectivas:} Pedagógica y Estratégica
	\item
	\textbf{Todos los alumnos están en algún grupo:} se comprueba que cada
	alumno pertenezca a un grupo. {Responsable:} Proveedor
	{Involucrados:} Diseñador {Perspectivas:} Tecnológica y
	Estratégica
\end{itemize}

\textbf{Realización:}

\begin{itemize}
	\item
	\textbf{El profesor responde en los foros dentro del límite de 48
		horas lectivas desde que se plantea la duda:} se comprueba que no
	hayan preguntas por parte de alumnos que estén sin responder en un
	tiempo razonable. {Responsable:} Facilitador
	{Involucrados:} Diseñador y Proveedor {Perspectivas:}
	Pedagógica y Tecnológica
	\item
	\textbf{Se ofrece retroalimentación de las tareas:} se comprueba que
	el profesor deje comentarios en la mayoría de calificaciones que haga.
	{Responsable:} Facilitador {Involucrados:} Diseñador y
	Proveedor {Perspectivas:} Pedagógica y Tecnológica
	\item
	\textbf{Las tareas están calificadas:} se comprueba que no hayan
	entregas de alumnos que hayan pasado una semana sin calificación.
	{Responsable:} Facilitador {Involucrados:} Diseñador y
	Proveedor {Perspectivas:} Pedagógica y Tecnológica
	\item
	\textbf{El calificador muestra cómo ponderan las diferentes tareas:}
	se comprueba que el calificador muestre los pesos de los items de
	calificación. {Responsable:} Facilitador {Involucrados:}
	Diseñador y Proveedor {Perspectivas:} Pedagógica y Tecnológica
	\item
	\textbf{Al menos un X \% de los alumnos responden a los cuesionarios:}
	se comprueba que hay un mínimo de alumnos que interactúan con los cuestionarios propuestos para el desarrollo del curso. {Responsable:} Proveedor {Involucrados:}
	Diseñador y Facilitador {Perspectivas:} Pedagógica
	\item
	\textbf{Al menos un X \% de los alumnos participa en los foros:}
	se comprueba que hay un mínimo de alumnos que interactúan con los foros propuestos para el desarrollo del curso. {Responsable:} Proveedor {Involucrados:}
	Diseñador y Facilitador {Perspectivas:} Pedagógica
\end{itemize}

\textbf{Evaluación:}

\begin{itemize}
	\item
	\textbf{La mayoría de alumnos responden a los feedbacks:} se comprueba
	que no hayan muchos alumnos que no respondan a los feedbacks.
	{Responsable:} Proveedor {Involucrados:} Diseñador y
	Facilitador {Perspectivas:} Pedagógica, Tecnológica, y
	Estratégica
	\item
	\textbf{Se utilizan encuestas de opinión:} se comprueba que el curso
	contenga encuestas de opinión. {Responsable:} Proveedor
	{Involucrados:} Diseñador y Facilitador {Perspectivas:}
	Pedagógica, Tecnológica, y Estratégica
\end{itemize}

